\chapter*{Forord}
Denne rapport er udarbejdet af gruppe B228, der er en gruppe af 1. semesters studerende på Byggeri og Anlæg-uddannelsen ved Aalborg Universitet. \textit{Model og Virkelighed} er det overordnede tema for projektet.
Projektet omhandler den nye cykel/gangbro ved Jernbanebroen, som går mellem Aalborg og Nørresundby. Aalborg kommune har bevilligt penge til broen, og projektet forventes færdigt indenfor de næste år.

Der rettes stor tak til vejleder Jannie Sønderkær Nielsen for vejledning og konstruktiv kritik. Endvidere rettes en stor tak til byrådsmedlem Jens Toft Nielsen for interview og rådgivning omkring den kommende Kulturbro, hvilken i dette projekt betegnes cykel/gangbro. Tak til personerne der har taget sig tid til at besvare spørgeskemaet. 
De samlede beregninger kan findes på hjemmesiden www.markhaurum.com, da der kun vil findes eksempler og korte uddrag af beregningerne i rapporten.

\textbf{Læsevejledning}

Der vil igennem rapporten fremtræde kildehenvisninger, og disse vil være samlet i en kildeliste bagerst i rapporten. Der er i rapporten anvendt kildehenvisning efter Harvardmetoden, så i teksten refereres en kilde med [Efternavn, År]. Denne henvisning fører til kildelisten, hvor bøger er angivet med forfatter, titel, udgave og forlag, mens Internetsider er angivet med forfatter, titel og dato. Figurer og tabeller er nummereret i henhold til kapitel, dvs. den første figur i kapitel 7 har nummer 7.1, den anden, nummer 7.2, osv. 

\phantom{Luft}

\phantom{Luft}

\begin{table}[H]
	\centering
		\begin{tabular}{c c c}
			\underline{\phantom{mmmmmmmmmmmmmm}} & \underline{\phantom{mmmmmmmmmmmmmm}} & \underline{\phantom{mmmmmmmmmmmmmm}} \\
			Jacob Scharling Jørgensen			& Karoline Vestergaard Hansen 		& Katrine Nørgaard Reberholt 			\\
			&&\\
			&&\\
			\underline{\phantom{mmmmmmmmmmmmmm}} & \underline{\phantom{mmmmmmmmmmmmmm}} & \underline{\phantom{mmmmmmmmmmmmmm}} \\
			Marc Lund Nielsen			& Michael Elgaard Mortensen 		& Morten Rask Jensen 				\\
			&&\\
			&&\\
			\underline{\phantom{mmmmmmmmmmmmmm}} \\
			Nikolaj Skov Gravesen						
		\end{tabular}
\end{table}