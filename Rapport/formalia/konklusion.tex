\chapter{Konklusion}
 Aalborg Kommune har gennem den nuværende kommuneplan en målsætning om, at blive Nordjyllands Vækstdynamo og være en by med fokus på udvikling af studerende, erhverv, kultur osv. Blandt et af kommunens fem fokuspunkter er “Aalborg - den attraktive storby”, som omhandler Vækstaksen, der beskriver et område i Aalborg, hvor der er stort fokus på byens udvikling og vækst. Derfor er det fordelagtigt at udvikle i netop dette område.
 \newline \indent{     }  De seneste 10 års udvikling på Aalborg Havnefront har givet øget fokus på udviklingen i netop dette område. Strøyberg Palæs beliggenhed ved Aalborg Havnefront, i den mest centrale del af Vækstaksen, gør, at tilbygningen har gode vilkår i forbindelse med vækst og udvikling, og tilbygningen kan blandt andet bruges til beboelse og erhvervslokaler, som kan tiltrække nye virksomheder til området. Dette vil være en lille del af Aalborgs udvikling, men en tilbygning til Strøybergs Palæ må dog formodes ikke at have en central betydning for Aalborgs udvikling, idet tilbygningen kun er en lille del af Vækstaksen.
 \newline \indent{     }  Strøybergs Palæ er underlagt Lokalplan 1-1-107, der beskriver to byggefelter til bygningen. I denne rapport tages der udgangspunkt i delområde B, hvor ny bebyggelse må opføres i 3 etager, samt en tagetage og kælder. Med udgangspunkt i Lokalplan 1-1-107 er bygningens størrelse og dimensioner bestemt. Hertil er der opstillet et statisk system, og der er valgt at indsætte tre stålrammer, hvor der dimensioneres efter den midterste ramme.
 \newline \indent{     }  Tilbygningens stålprofiler er dimensioneret ud fra ståltype S235 med profil nr. 450, samt ud fra de permanente og variable laster der virker på tilbygningen; egenlast, jordlast, vindlast, snelast og nyttelast. Herudfra er tilbygningens brud- og anvendelsesgrænsetilstand bestemt.
 \newline \indent{     }  Ud fra spændingstilstanden kan det konkluderes, at tilbygningen har en tilstrækkelig bæreevne, idet der fås en maksimal spænding på 100,51 MPa mod en flydespænding på 204,54 MPa. 
 \newline \indent{     }  For anvendelsesgrænsen er udbøjningen bestemt for tre af konstruktionens stålstænger, og her er den vandrette udbøjning bestemt til 1.74 m, mens den lodrette udbøjning er bestemt til 0,5 m, og denne overskrider derfor de anbefalede værdier for udbøjning af bærende konstruktioner. 
 \newline \indent{     }  Tilbygningens fundament er dimensioneret ud fra Aalborgs geologi. Aalborgs undergrund er primært bestående af Aalborgler og funderingen til Strøybergs Palæ burde dermed være pælefundering. I dette projekt ønskes der dog, at blive arbejdet med direkte fundering, og derfor anvendes der boreprofiler fra Hals, hvor undergrunden primært består af senglacialt sand, hvilket antages at være boreprofilerne fra området ved Strøybergs Palæ.
 \newline \indent{     }  De udførte laboratorieforsøg, hvor formålet er at bestemme friktionsvinklen, er udført på baskarpsand fra Sverige, der antages at være sandet fra boreprofilerne. Ud fra de fire forsøg er friktionsvinklen skønnet til 32,33 grader, der anvendes i videre beregninger af fundamentets areal. Der laves et punktfundament for hver søjle i det statiske system, med en længde og bredde på 1 m.