\chapter{Konklusion}
I dag er Limfjordsbroen den eneste overgang for fodgængere og cyklister over Limfjorden. Det har dog gennem længere tid været på tale, at konstruere en cykel/gangbro på Jernbanebroen over Limfjorden, således at de bløde trafikanter får endnu en overgang over Limfjorden. Dertil har projektet, med udgangspunkt i problemformuleringen og de dertilhørende problemstillinger, undersøgt den nye cykel/gangbro ud fra et kontekstuelt og teknisk perspektiv. 
\newline
\newline
Projektgruppen kan på baggrund af de indsamlede data og interviews konkludere, at baggrunden for at konstruere en cykel/gangbro, ligger i byens udvikling og ikke i lige så høj grad for nødvendigheden af broen på nuværende tidspunkt. Der higes ikke efter en ny overgang, men der konkluderes at en sådan overgang vil blive taget godt imod, da den vil forbedre infrastrukturen imellem Aalborg og Nørresundby, og dermed gøre det lettere for enkelte målgrupper.Ydermere kan det konkluderes, at de færreste er direkte imod en ny overgang. Som skrevet, opføres cykel/gangbroen på grund af Aalborgs udvikling, med de nye havnefronter og en kommende letbane, som broen også skaber muligheder for.
\newline
\newline
Der har været flere løsningsforslag til hvordan cykel/gangbroen skal se ud, hvilket materiale den skal fremstilles i og hvordan den skal dimensioneres. Projektet har belyst alle tre løsningsforslag på forskellig vis. 
\newline \indent{     }  Gennem beregninger af stangkræfterne i Jernbanebroens gitterkonstruktion er det første løsningsforslag blevet undersøgt, for at undersøge om Jernbanebroen vil kunne holde til denne cykel/gangbro og den ekstra last, som Jernbanebroen pålægges med en ny cykel/gangbro monteret på siden. Her er spændingerne beregnet for alle stængerne, og det kan konkluderes, at flere af stængernes spændinger overskrider den regningsmæssige flydespænding, endda med op til 30\%, og derfor vil dette løsningsforslag ikke kunne lade sig gøre med de dimensioner og antagelser, som projektgruppen har gjort sig omkring denne løsning. Der vil opstå brud på stængerne, og der vil være en stor risiko for, at stængerne vil knække sammen. 
\newline \indent{     }  Alle tre løsningsforslag er efterfølgende behandlet i en diskussion, hvor deres fordele og ulemper er opstillet. Aalborg Kommune har et samlet budget på 25 mio. til en ny cykel/gangbro, men prisen er kun en ud af mange faktorer, som skal gå op. Jernbanebroens egen- og nyttelast gør, at Jernbanebroen allerede i dag er belastet med 95\% af sin maksimale bæreevne, og derfor har vægten en stor betydning for løsningsforslaget, og her kan komposit blive afgørende, da det er meget lettere end stål. Samtidig er stålprojektet begrænset til en maksimal bredde på 1,8 meter, mens løsningsforslagene, hvor kompositmateriale indgår, gør en bredde på 2,0 meter mulig. 
