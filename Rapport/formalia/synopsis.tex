Dette P2-projekt omhandler en tilbygning til Strøybergs Palæ.
\newline
\newline
Strøybergs Palæ er beliggende ved havnen i Aalborg inden for Vækstaksen fremsat af Aalborg Kommune. Bygningen skal laves henfør lokalplan 1-1-107. Projektet beskriver tilbygningen ud fra to perspektiver. En kontekstuel del, hvor der lægges fokus på Aalborgs udvikling gennem tiden, og dens nuværende og planlagte udvikling, heriblandt Vækstaksen. Der vil blive redegjort for disse og diskuteret, om den planlagte udvikling overhovedet er realistisk. Den tekniske del vil have fokus på dimensioneringen af tilbygningen og dens fundament, hvor der opstilles et statisk system samt en række  permanente og variable laster, som tilbygningen skal dimensioneres ud fra, og den endelige ståltype og stålprofil bestemmes ud fra gennem beregninger for anvendelses- og brudgrænsetilstanden. Der vil ligeledes blive redegjort for Aalborgs geologi og forskellige jordarters udseende og styrke. Ud fra dette vil der blive angivet en fundamenttype og udregnet størrelsen af det, så fundamentet har tilstrækkelig bæreevne til dimensioneringen af tilbygningen.
\newline \indent{     }  Gennem beregningerne for brudgrænsetilstanden er det bestemt, at ståltypen S235 og profil nr. 450 har en tilstrækkelig styrke, og at det statiske system for konstruktionen kan holde til lasterne. Dog er udbøjningerne gennem anvendelsesgrænsetilstanden bestemt til at overskride de acceptable værdier, og kan derfor ikke godtages. RET EVT TIL!