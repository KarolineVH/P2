Dette P1-projekt omhandler den nye cykel/gangbro, som skal monteres på Jernbanebroen ved Limfjorden i Aalborg.
\newline
\newline
Cykel/gangbroen har været undervejs længe, og der er gennem de seneste seks år fremstillet tre løsningsforslag. Hertil beskriver projektet den kommende cykel/gangbro ud fra to perspektiver; en kontekstuel del, hvor fokus vil være på baggrunden for cykel/gangbroen, samt holdningen til denne. Der er foretaget et interview med formanden for projektet, Jens Toft Nielsen og et spørgeskema er uddelt, for at undersøge holdningen for denne. Den tekniske del fokuserer på bæreevnen for Jernbanebroens stænger, da Jernbanebroens egen- og nyttelast udgør 95\% af den maksimale bæreevne. Derfor er dimensioneringen af cykel/gangbroen begrænset, og projektet vil undersøge, hvorfor cykel/gangbroen vil dimensioneres således, samt hvor meget cykel/gangbroen vil øge belastningen for Jernbanebroens stænger i gitterkonstruktionen. Dette gøres ud fra knudepunktsmetoden, hvor beregninger er vedlagt som bilag.
