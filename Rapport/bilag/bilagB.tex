\chapter{Bilag B}
\section{Sigteanalyse}
\textbf{Formål}
\newline
Formålet med forsøger er at bestemme jordkornenes vægtmæssige fordeling efter størrelse i sand- og grusfraktion, for at beregne uensformighedstallet for jorden, der skal bruges i de videre beregninger af friktionsvinklen. 
\newline
\newline
Velsorteret: U<2
\newline 
Sorteret: 2<U<3,5
\newline
Ringe sorteret: 3,5<U<7
\newline
Usorteret: U>7
\newline
\newline
\textbf{Apparaturliste}
\begin{itemize}
	\item[-] Sigter med mindst maskevidde på 0,063 mm
	\item[-] Rystemaskine
	\item[-] Vægt med vejenøjagtighed på 0,01 g
	\item[-] Sigtebørste
	\item[-] Skåle i korrosion bestandigt materiale
\end{itemize}
\textbf{Fremgangsmåde}
\newline
Ved en sigteanalyse kan der både udføres en grovsigtning og en finsigtning. Grovsigtningen udføres, hvis materialet vurderes til at have partikler over 16 mm. I dette forsøg er der kun udført en finsigtning, da partiklerne vurderes til at være mindre en 16 mm. Sigtningen er udført på sigter fra 0,063 mm til 2 mm (sigtemålene kan ses i tabellen over resultaterne). 
\newline \indent{     }   Først rengøres hver enkelt sigte forsigtigt med sigtebørste, og herefter samles sigterne forløbende fra den største maskevdde øverst til bunden nederst. Det afmålte materiale hældes på sigten med den største maskevidde på 2 mm, hvorefter sigtetårnet placeres i rystemaskinen og sigtes i 20 minutter.
\newline \indent{     }   Sigteresterne i hver enkelt sigte overføres til skåle og vejes. Hver sigte placeres med bunden opad på et stort stykke papir, og der fejes let på bagsiden, således materialet der sidder fast i maskerne løsnes.
\newline \indent{     }   Alle resultaterne skrives ind i nedenstående tabel, hvor det procentvise gennemfald i hver sigte beregnes ved $gennemfald [\%] = \frac{gennemfald [g]}{samlet prøve [g]}\cdot 100 [\%]$. Herefter optegnes en sigtekurve over resultaterne med det procentvise gennemfald af y-aksen i aritmisk skala, og kornstørrelse af x-aksen i logaritmisk skala. Herpå aflæses de to punkter der ligger mellem henholdsvis 10\% og 60\%, og der laves lineær regression imellem de to punkter. Herudfra kan 10\%-fraktilen, $d_{10}$, og 60\%-fraktilen, $d_{60}$, beregnes. Disse bruges til at udregne uensformighedstallet $U = \frac{d_{60}}{d_{10}}$
\newline
\newline
\textbf{Resultater}
\newline
Tabel over resultater for det første udførte forsøg:
\newline
\newline
Tabel over resultater for det andet udførte forsøg:
\newline
\newline
\textbf{Beregninger}
\newline
\underline{Forsøg 1}
\newline
\newline
HER INDSÆTTES SIGTEKURVE FOR FORSØG 1
\newline
\newline
For at finde 10\%-fraktilen er der lavet lineær regression imellem sigte med maskestørrelse 0,075 mm og 0,125 mm, hvor følgende ligning fremgår: 

\begin{center}
	$y=862.42\cdot x - 63.223$
\end{center}

For at finde 60\%-fraktilen er der lavet lineær regression imellem sigte med maskestørrelse 0,125 mm og 0,15 mm, hvor følgende ligning fremgår:

\begin{center}
	$y=937.12\cdot x - 72.56$
\end{center}

Henholdsvis 10\%-fraktilen og 60\%-fraktilen er beregnet til:



\underline{Forsøg 2}
\newline
\newline
HER INDSÆTTES SIGTEKURVE FOR FORSØG 2





