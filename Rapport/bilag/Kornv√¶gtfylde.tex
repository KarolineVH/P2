\Chapter{Bilag C}
\textbf{formål}
\newline
Formålet med forsøget er, at finde den relative densitet $d_{s} $, også kaldet kornvægtsfylden for en jordprøve. 
\newline
\newline
\textbf{Apparaturliste]
\begin{itemize}
\item[-] Pyknometer
\item[-] Bægerglas
\item[-] Termometer, nøjagtighed 0,1^{\circ}C
\item[-]Vægt, vejenøjagtighed på 0,001g
\item[-] Tørreskab, temperatur til 105^{\circ}C
\end{itemize}
\newline
\newline
\textbf{Fremgangsmåde]
\newline
Dette forsøg laves med friktionsjord (tør metode).
Der udtages en prøve af 150g tørstof og dette placeres i et 500mL pyknometer. Pyknometeret fyldes til ca. halvt med luftfrit demineraliseret vand og der drejes på pyknometeret for at undgå luftbobler. Derefter hældes der så meget vand i, at vandet flyder over, når proppen sættes i. Proppen sættes i, og der sørges igen for at der ikke er luftbobler til stede. Pyknometeret med materialet, vandet og prop vejes og kaldes $W_{1}$
Derefter måles temperaturen i pyknometeret og noteres. Så aflæses $W_{2}$
 i et kalibreringsskema og noteres. Der benyttes linear interpolation til at finde vægten af pyknomeret ed temperaturer, som ikke er ens med dem opgivet i skemaet. 
 
 \newline
 \newine
 \textbf{Resultater}
 \begin{center}
 	\begin{tabular
 		
 		
 		
 		
 		
 		
 		
 		
 	\end{tabular}
 \end{center}

\newline
\newline
\textbf{Beregninger}

Rumfang af tørstof findes ved formlen 

$(W_{s}+W_{2}-W_{1})/p_{w}^t$


Alle tallene kendes og rumfanget kan beregnes for forsøg 1 og forsøg 2.

Forsøg 1

$(161.27g+641.164g-728.29g)/0.99780g/mL)=7,4*10^-5m^3$

Dette omregnes til mm^3
\newline
$(7,4*10^-5m^3*1000000000mm^3)=74307,48mm^3$

Altså er tørstoffet for forsøg 1 = 74307,48mm^3
\newline
Forsøg 2:

$(150,06g+615,967g-709,40g)/0.99757g/mL)=5,7*10^-5m^3$

Dette omregnes til mm^3
\newline

$(5,7*10^-5m^3*1000000000mm^3)=56764,94mm^3$

\newline
Den relative densitet findes ved formlen 

$d_{s}$$=(W_{s}*p_{w}^t)/W_{s}+W_{2}-W_{1}$

Den relative densitet findes for begge forsøgs resultater:
\newline
Forsøg 1:
\newline
$(161,27g*0,99780g/mL)/161,27g+641,164g-728,89g=2,19g/m^3$
\newline
Forsøg 2
\newline
$(150,06g*0,99757g/mL)/150,06g+615,967g-709,40g=2,64g/m^3$


\textbf{Fejlkilder}
\newline
En fejlkilde ved dette forsøg, er luftbobler i pyknometeret. Derudover var det en del af forsøget af benytte en vacuumekssikkator, men pga. manglende tid blev dette trin sprunget over. En anden fejlkilde er at der kan være forskellige temperature i pyknometeret og udover disse fejlkilder er der også måleusikkerheder, som eksempelvis når der skulle vejes tørstof og pyknometer eller temperaturen.

