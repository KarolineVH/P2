\chapter{Forsøg: Vandindhold}

\textbf{Formål}
\newline
Formålet med dette forsøg er at finde vandindholdet, w, i en jordprøve. Vandindholdetn er defineret som, jordens vægttab i \% af tørvægten ved tørring i et varmeskab ved en temperatur på 105$^{\circ}$C til konstant vægt er opnået.
\newline
\newline
\textbf{Apperaturliste}
\begin{itemize}
\item[-] Vægt, vejenøjagtighed $0,\!01$ g
\item[-] Skål i varme- og korrosionsbestandigt materiale
\item[-] Tørreskab, temperatur til 105$^{\circ}$C
\end{itemize}

\begin{center}
	\begin{tabular}{ |c|c| } 
		\hline
		Partikel diameter $D_{90}$ [mm] & Minimum mængde af våd prøve [g] \\	\hline
		1,0 & 25 \\		\hline
		2,0 & 100 \\	\hline 
		4,0 & 300 \\ 	\hline
		16,0 & 500 \\	\hline
		31,5 & 1500 \\	\hline
		63,0 & 5000 \\	\hline	
	\end{tabular}
\end{center}

\textbf{Fremgangsmåde}
\newline
Først findes en ren og tør foliebakke vejes, og vægten noteres som \textit{sk.} Efterfølgende udtages en passsende mængde jord, jf. tabel 1, og anbringes i foliebakken, og det hele vejes omgående sammen, og noteres som \textit{W+sk}.
\newline
Foliebakken anbringes nu i tørreskabet ved 105$^{\circ}$C, og tørres fra fredag til mandag (Normalt tørres det i 24 timer, da konstant vægt normalt er opnået eftter dette tidsrum, hvilket også gøres her). Efter tørringen sættes foliebakken til afkåling i vacuumekssikator til rumtemperatur er opnået. Den afkølede foliebakke med den nu tørre jordprøve vejes \textit{$W_{s}+sk$}.
\newline
\newline
\textbf{Beregninger}
\newline
Beregninger udføres på følgende måde:
\begin{center}
	$w=\frac{W_w}{W_s}\cdot100\%=\frac{(W + sk) - (W_s + sk)}{(W_s + sk) - sk}\cdot100\%$
\end{center}

Vandindholdet regnes nu forsøg 1:
\begin{center}
	$w=\frac{(81,\!02)-(80,\!99)}{(80,\!99)-3,\!07}\cdot100\%=0,\!04\%$
\end{center}

Vandindholdet regnes nu for forsøg 2:
\begin{center}
	$w=\frac{(89,\!83)-(89,\!79)}{(89,\!79)-3,\!11}\cdot100\%=0,\!05\%$
\end{center}

\textbf{Resultater}
\begin{center}
	\begin{tabular}{ |c|c|c| } 
		\hline
		Prøve nr. & 1 & 2 \\	\hline
		W+sk [g] & 81,02 & 89,83 \\	\hline
		$W_s$+sk [g] & 80,99 & 89,79 \\	\hline 
		sk [g] & 3,07 & 3,11 \\ \hline
		$W_w=(W+sk) - (W_s+sk)$ [g] & 0,03 & 0,04 \\	\hline
		$W_s=(W_s+sk) - sk$ [g] & 77,92 & 86,68 \\	\hline		
		w [\%] & 0,04\% & 0,05\% \\	\hline	
	\end{tabular}
\end{center}

\textbf{Fejlkilder}
\newline
Forskelligheden i de 2 prøveforsøg kan skyldes brugen af 2 forskellige vægte med forskellige størrelse usikkerheder. En anden fejlkilde er at størrelserne på prøverne er forskellige vægten er med en difference på $8,\!81$ g, samt der ikke er taget højde for tabel 1, så mængden af prøvemateriale sker ud fra anbefaling.
\newline
\newline
\textbf{Delkonklusion}
\newline
Udfra de opnåede resultater kan det konkluderes at 