\chapter{Forsøg: Vandindhold}

\textbf{Apparaturliste}
\begin{itemize}
\item[-] Vægt med vejenøjagtighed $0,\!01$ g
\item[-] Skål i varme- og korrosionsbestandigt materiale
\item[-] Tørreskab med temperatur til 105$^{\circ}$C
\end{itemize}

\textbf{Fremgangsmåde}
\newline
Først findes en ren og tør foliebakke som vejes, og vægten noteres som \textit{sk.} Efterfølgende udtages en passende mængde jord, jf. ovenstående tabel. Dette anbringes i foliebakken, og det hele vejes omgående sammen, og noteres som \textit{W+sk}.
\newline \indent{     }  Foliebakken anbringes nu i tørreskabet ved 105$^{\circ}$C, og tørres fra fredag til mandag (normalt tørres det i 24 timer, da konstant vægt normalt er opnået efter dette tidsrum, hvilket også gøres her). Efter tørringen sættes foliebakken til afkøling i vacuumekssikator til rumtemperatur er opnået. Den afkølede foliebakke med den nu tørre jordprøve vejes \textit{$W_{s}+sk$}.
\newline
\newline
\textbf{Resultater}
\begin{table} [H]
\begin{center}
	\begin{tabular}{ c c c c c c c} 
		\hline
		Nr. & $W+sk$ & $W_s+sk$ & $sk$ & $W_w$ & $W_s$ & $W$ \\	
		& & & & $(W+sk) - (W_s+sk)$ & $(W_s+sk) - sk$ & \\
		& [g] & [g] & [g] & [g] & [g] & [\%] \\ \hline
		1 & 81,02 & 80,99 & 3,07 & 0,03 & 77,92 & 0,04 \\
		2 & 89,83 & 89,79 & 3,11 & 0,04 & 86,68 & 0,05
	\end{tabular}
	\caption{Resultater for vandindhold}
	\label{tab:bilaga1}
\end{center}
\end{table}

\textbf{Fejlkilder}
\newline
Forskelligheden i de to prøveforsøg kan skyldes brugen af to forskellige vægte med forskellige størrelse usikkerheder. Andre fejlkilder er, at størrelserne på prøverne er forskellige, vægten $W+sk$ har en difference på $8,\!81$ g, samt der ikke er taget højde for tabel 1, så mængden af prøvemateriale sker ud fra anbefaling.