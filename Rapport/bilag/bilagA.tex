\chapter{Forsøg: Vandindhold}

\textbf{Formål}
\newline
Formålet med dette forsøg er at finde vandindholdet, w, i en jordprøve. Vandindholdetn er defineret som, jordens vægttab i \% af tørvægten ved tørring i et varmeskab ved en temperatur på 105$^{\circ}$C til konstant vægt er opnået.
\newline
\newline
\textbf{Apperaturliste}
\begin{itemize}
\item[-] Vægt, vejenøjagtighed 0.01 g
\item[-] Skål i varme- og korrosionsbestandigt materiale
\item[-] Tørreskab, temperatur til 105$^{\circ}$C
\end{itemize}

\textit{Tabel 1}

%Tabel skal være her!!%

\textbf{Fremgangsmåde}
\newline
Først findes en ren og tør foliebakke vejes, og vægten noteres som \textit{sk.} Efterfølgende udtages en passsende mængde jord, jf. tabel 1, og anbringes i foliebakken, og det hele vejes omgående sammen, og noteres som \textit{W+sk}.
\newline
Foliebakken anbringes nu i tørreskabet ved 105$^{\circ}$C, og tørres fra fredag til mandag (Normalt tørres det i 24 timer, da konstant vægt normalt er opnået eftter dette tidsrum). Efter tørringen sættes foliebakken til afkåling i vacuumekssikator til rumtemperatur er opnået. Den afkølede foliebakke med den nu tørre jordprøve vejes,\textit{$W_{s}+sk$}


%Resultat tabel indsættes%

\textbf{Beregninger}
\newline
\textit{Beregninger udføres på følgende måde:}
\newline
\begin{center}
$w=\frac{$W_{w}$}{$W_{s}$}*100\%=$
