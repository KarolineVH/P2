\chapter{Forsøg: Vandindhold}

\textbf{Apparaturliste}
\begin{itemize}
\item[-] Vægt med vejenøjagtighed $0,\!01$ g
\item[-] Skål i varme- og korrosionsbestandigt materiale
\item[-] Tørreskab med temperatur til 105$^{\circ}$C
\end{itemize}

\begin{center}
	\begin{tabular}{ |c|c| } 
		\hline
		Partikel diameter $D_{90}$ [mm] & Minimum mængde af våd prøve [g] \\	\hline
		1,0 & 25 \\		\hline
		2,0 & 100 \\	\hline 
		4,0 & 300 \\ 	\hline
		16,0 & 500 \\	\hline
		31,5 & 1500 \\	\hline
		63,0 & 5000 \\	\hline	
	\end{tabular}
\end{center}

\textbf{Fremgangsmåde}
\newline
Først findes en ren og tør foliebakke som vejes, og vægten noteres som \textit{sk.} Efterfølgende udtages en passende mængde jord, jf. ovenstående tabel. Dette anbringes i foliebakken, og det hele vejes omgående sammen, og noteres som \textit{W+sk}.
\newline \indent{     }  Foliebakken anbringes nu i tørreskabet ved 105$^{\circ}$C, og tørres fra fredag til mandag (normalt tørres det i 24 timer, da konstant vægt normalt er opnået efter dette tidsrum, hvilket også gøres her). Efter tørringen sættes foliebakken til afkøling i vacuumekssikator til rumtemperatur er opnået. Den afkølede foliebakke med den nu tørre jordprøve vejes \textit{$W_{s}+sk$}.
\newline
\newline
\textbf{Resultater}
\begin{center}
	\begin{tabular}{ |c|c|c| } 
		\hline
		Prøve nr. & 1 & 2 \\	\hline
		W+sk [g] & 81,02 & 89,83 \\	\hline
		$W_s$+sk [g] & 80,99 & 89,79 \\	\hline 
		sk [g] & 3,07 & 3,11 \\ \hline
		$W_w=(W+sk) - (W_s+sk)$ [g] & 0,03 & 0,04 \\	\hline
		$W_s=(W_s+sk) - sk$ [g] & 77,92 & 86,68 \\	\hline		
		w [\%] & 0,04\% & 0,05\% \\	\hline	
	\end{tabular}
\end{center}

\textbf{Fejlkilder}
\newline
Forskelligheden i de to prøveforsøg kan skyldes brugen af to forskellige vægte med forskellige størrelse usikkerheder. Andre fejlkilder er, at størrelserne på prøverne er forskellige, vægten $W+sk$ har en difference på $8,\!81$ g, samt der ikke er taget højde for tabel 1, så mængden af prøvemateriale sker ud fra anbefaling.
\newline
\newline
\textbf{Delkonklusion}
\newline
Udfra de opnåede resultater kan det konkluderes at det benyttede materiale vurderes at være tørt og det meget lille vandindhold ikke har indflydelse på de øvrige resultater.