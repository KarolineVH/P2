\chapter{Forsøg: Kornvægtfylde}
\textbf{Formål}
\newline
Formålet med forsøget er, at finde den relative densitet $d_{s}$, også kaldet kornvægtfylden for en jordprøve. 
\newline
\newline
\textbf{Apparaturliste}
\begin{itemize}
\item[-] Pyknometer
\item[-] Bægerglas
\item[-] Termometer, nøjagtighed $0,1^{\circ}$C
\item[-] Vægt, vejenøjagtighed på 0,001 g
\item[-] Tørreskab, temperatur til 105$^{\circ}$C
\end{itemize}

\textbf{Fremgangsmåde}
\newline
Dette forsøg laves med friktionsjord (tør metode). Der udtages en prøve af 150 g tørstof og dette placeres i et 500 mL pyknometer. Pyknometeret fyldes ca. halvt op med luftfrit demineraliseret vand og der drejes på pyknometeret for at undgå luftbobler. Derefter hældes der så meget vand i, at vandet flyder over, når proppen sættes i. Proppen sættes i, og der sørges igen for, at der ikke er luftbobler til stede. Pyknometeret med materiale, vand og prop vejes og kaldes $W_{1}$. Herefter måles temperaturen i pyknometeret, og temperaturen noteres. Så aflæses $W_{2}$ i et kalibreringsskema og noteres. Der benyttes linear interpolation til at finde vægten af pyknomeret ved temperaturer, som ikke er ens med dem opgivet i skemaet. 
\newline
\newline
\textbf{Resultater}
\begin{center}
	\begin{tabular}{ |c|c|c| } 
		\hline
		 & Forsøg 1 & Forsøg 2 \\	\hline
		Pyknometer nr. & 103 & 100 \\	\hline
		$W_1=W_{pyk} + W_s + W_{vand}$ [g] & 728,89 & 709,40 \\	\hline 
		Temperatur [$^{circ}$C] & 22 & 23 \\ \hline
		$W_2=W_{pyk+vand}$ [g] & 641,164 & 615,967 \\	\hline
		Tørstof, $W_s$, [g] & 161,27 & 150,06 \\	\hline
		Vands densitet, $p_{w}^t$, [$\frac{g}{mL}$] & 0,998 & 0,998 \\	\hline
		Relativ densitet, $d_s=\frac{W_s\cdot p_{w}^t}{W_s + W_2 - W_1}$, [$\frac{g}{m^3}$] & 2,188 & 2,644		
	\end{tabular}
\end{center}

\textbf{Beregninger}
\newline
Rumfang af tørstof findes ved formlen:
\begin{center}
	$\frac{W_s + W_2 - W_1}{p_{w}^t$
\end{center}

Alle tallene kendes og rumfanget kan beregnes for forsøg 1 og forsøg 2.
\newline
\underline{Forsøg 1}
\begin{center}
	$\frac{161,27 g +641,164 g - 728,29 g}{0,998 \frac{g}{mL}}=7,431\cdot 10^5 m^3$
\end{center}