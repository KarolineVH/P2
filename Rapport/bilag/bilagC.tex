\chapter{Forsøg: Kornvægtfylde}

\textbf{Apparaturliste}
\begin{itemize}
\item[-] Pyknometer
\item[-] Bægerglas
\item[-] Termometer med nøjagtighed $0,\!1^{\circ}$C
\item[-] Vægt med vejenøjagtighed på 0,001 g
\item[-] Tørreskab, temperatur til 105$^{\circ}$C
\end{itemize}

\textbf{Fremgangsmåde}
\newline
Dette forsøg laves med friktionsjord (tør metode). Der udtages en prøve af 150 g tørstof, og dette placeres i et 500 mL pyknometer. Pyknometeret fyldes ca. halvt op med luftfrit demineraliseret vand, og der drejes på pyknometeret for at undgå luftbobler. Derefter hældes der vand i, indtil vandet flyder over, når proppen sættes i. Proppen sættes i, og der sørges igen for, at der ikke er luftbobler til stede. Pyknometeret med materiale, vand og prop vejes og kaldes $W_{1}$, og beregnes altså ved $W_1 = W_{pyk} + W_s + W_{vand}$. Herefter måles temperaturen i pyknometeret, og temperaturen noteres. Dernæst aflæses $W_{2}$ i et kalibreringsskema og noteres $p_w^t$.
\newline
\newline
\textbf{Resultater}
\begin{table}
\begin{center}
	\begin{tabular}{ |c|c|c| } 
		\hline
		 & Forsøg 1 & Forsøg 2 \\	\hline
		Pyknometer nr. & 103 & 100 \\	\hline
		$W_1$ [g] & 728,89 & 709,40 \\	\hline 
		Temperatur [$^{\circ}$C] & 22 & 23 \\ \hline
		$W_2$ [g] & 641,164 & 615,967 \\	\hline
		$W_s$ [g] & 161,27 & 150,06 \\	\hline
		$\rho_{w}^t$ [$\frac{g}{mL}$] & 0,998 & 0,998 \\	\hline
		$d_s$ [$\frac{g}{m^3}$] & 2,188 & 2,644	\\	\hline	
	\end{tabular}
	\caption{Resultater for kornvægtfylde}
	\label{tab:bilagc1}
\end{center}
\end{table}

\textbf{Fejlkilder}
\newline
En fejlkilde ved dette forsøg er luftbobler i pyknometeret. Derudover var det en del af forsøget at benytte en vacuumekssikator, men grundet manglende tid blev dette trin sprunget over. En anden fejlkilde er, at der kan være forskellige temperaturer i pyknometeret, og udover disse fejlkilder er der også måleusikkerheder, som eksempelvis når der tørstof og pyknometer skulle vejes eller temperaturen skulle måles. 
Ved forsøg 1 blev der vejet forkert og dette antages at være årsagen til det misvisende resultat.