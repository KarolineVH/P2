\Chapter{Bilag C}
\textbf{formål}
\newline
Formålet med forsøget er, at finde den relative densitet $d_{s} $, også kaldet kornvægtsfylden for en jordprøve. 
\newline
\newline
\textbf{Apparaturliste]
\begin{itemize}
\item[-] Pyknometer
\item[-] Bægerglas
\item[-] Termometer, nøjagtighed 0,1^{\circ}C
\item[-]Vægt, vejenøjagtighed på 0,001g
\item[-] Tørreskab, temperatur til 105^{\circ}C
\end{itemize}
\newline
\newline
\textbf{Fremgangsmåde]
\newline
Dette forsøg laves med friktionsjord (tør metode).
Der udtages en prøve af 150g tørstof og dette placeres i et 500mL pyknometer. Pyknometeret fyldes til ca. halvt med luftfrit demineraliseret vand og der drejes på pyknometeret for at undgå luftbobler. Derefter hældes der så meget vand i, at vandet flyder over, når proppen sættes i. Proppen sættes i, og der sørges igen for at der ikke er luftbobler til stede. Pyknometeret med materialet, vandet og prop vejes og kaldes $W_{1}$
Derefter måles temperaturen i pyknometeret og noteres. Så aflæses $W_{2}$
 i et kalibreringsskema og noteres. Der benyttes linear interpolation til at finde vægten af pyknomeret ed temperaturer, som ikke er ens med dem opgivet i skemaet. 
 
 \newline
 \newine
 \textbf{Resultater}
 \begin{center}
 	\begin{tabular}{|p{5cm}|p{5cm}|p{5cm}
 		\hline
 		
 		
 		
 		
 		
 		
 		
 		
 	\end{tabular}
 \end{center}





