\chapter{Beregning af peakhastighedstrykket}
Nedenfor gives et beregningseksempel af peakhastighedstrykket, $q_p$, for taget med vindretning fra vest. Fremgangsmåden for vindretningerne øst og nord er identisk med nedenstående eksempel, blot med andre værdier fra de samme kilder som anvendes i eksemplet, da vindretning og højden ændres.
\newline
\newline
Den maksimale belastning fra vinden, peakhastighedstrykket $q_p$, bestemmes ved:
\begin{equation}
	q_p(z_e)=[1+7I_v(z_e)]\frac{1}{2}pv_m^2(z_e)
\end{equation}
\begin{itemize}
	\item[-] $I_v$: Vindturbulens
	\item[-] $\rho$: Densiteten for luft ved $20^{\circ}$, $1,\!25 \frac{\text{kg}}{\text{m}^3}$
	\item[-] $v_m$: Middelvindhastigheden
\end{itemize}
For at bestemme peakhastigheden, beregnes først vindturbulens $I_v(z)$ samt middelvindhastigheden $v_m$.
\newline
\newline
Vindturbulens, $I_v(z)$, bestemmes ved:
\begin{equation}
	I_v(z)=\frac{\sigma_v}{V_m(z)}=\frac{k_1}{c_0(z) \text{ln}(\frac{z}{z_0})}
\end{equation}
\begin{itemize}
	\item[-] $k_1$: Turbulensfaktor, sættes til $1,\!0$ \citep[ kapitel 4.4]{EU91}
	\item[-] $c_0(z)$: Orografifaktoren, som sættes til $1,\!0$ \citep[ kapitel 4.3.1]{EU91}
	\item[-] $z$: Højden, som med taget er 19 m
	\item[-] $z_0$: Ruhedslængde, som sættes til $1,\!0$ for terrænkategori IV \citep[ tabel 4.1 kapitel 4.3.2]{EU91}
\end{itemize}
Vindturbulensen bestemmes til:
\begin{equation}
	I_v(z)=\frac{1,\!0}{1,\!0\cdot \text{ln}(\frac{19}{1,0})}=0,\!34
\end{equation}

Middelvindhastigheden, $v_m$, bestemmes ved:
\begin{equation}
	v_m(z)=c_r(z)c_0(z)v_b
\end{equation}

\begin{itemize}
	\item[-] $c_r(z)$: Ruhedsfaktor
	\item[-] $v_b$: Basisvindhastigheden
\end{itemize}
Til at bestemme middelvindhastigheden, beregnes basisvindhastigheden samt ruhedsfaktoren.
\newline
\newline
Basisvindhastigheden, $v_b$, bestemmes ved:
\begin{equation}
	v_b=c_{dir}c_{season}v_{b,0}
\end{equation}
\begin{itemize}
	\item[-] $c_{dir}$: Retningsfaktor, som sættes til $1,\!0$ ved vind fra vest \citep[ tabel 1a kapitel 4.2]{EU91}
	\item[-] $c_{season}$: Årstidsfaktor, som sættes til $1,\!0$ \citep[ tabel 1b kapitel 4.2]{EU91}
	\item[-] $v_{b,0}$: Grundværdi for basisvindhastigheden, som sættes til 24 $\frac{\text{m}}{\text{s}}$ \citep[ kapitel 4.2]{EU91}
\end{itemize}
Basisvindhastigheden bestemmes til:
\begin{equation}
	v_b=1,\!0\cdot 1,\!0\cdot 24 \frac{\text{m}}{\text{s}}=24 \frac{\text{m}}{\text{s}}
\end{equation}

Ruhedsfaktor, $c_r(z)$, bestemmes ved:
\begin{equation}
	c_r(z)=k_r \text{ln}(\frac{z}{z_0})
\end{equation}
\begin{itemize}
	\item[-] $k_r$: Terrænfaktor
\end{itemize}

Terrænfaktoren, $k_r$, bestemmes ved:
\begin{equation}
	k_r=0,\!19\cdot (\frac{z_0}{z_{0,II}})^{0,07}
\end{equation}

\begin{itemize}
	\item[-] $z_{0,II}$: Værdi for ruhedslængde for terrænkategori II, som sættes til $0,\!05$ \citep[ kapitel 4.3.2]{EU91}
\end{itemize}

\begin{equation}
	k_r=0,\!19\cdot (\frac{1,\!0}{0,\!05})^{0,07}=0,\!234
\end{equation}
Ruhedsfaktor bestemmes til:
\begin{equation}
	c_r(z)=0,\!234\cdot \text{ln}(\frac{19}{1,\!0})=0,\!690
\end{equation}
Middelvindhastigheden bestemmes til:
\begin{equation}
	v_m(z)=0,\!690\cdot 1,\!0\cdot 24 \frac{\text{m}}{\text{s}}=16,\!569 \frac{\text{m}}{\text{s}}
\end{equation}
Peakhastighedstrykket $q_p$ i højden z, bestemmes til:
\begin{equation}
	q_p(z_e)=[1+7\cdot 0,\!340]\cdot \frac{1}{2}\cdot 1,\!25 \frac{\text{kg}}{\text{m}^3}\cdot (16,\!569 \frac{\text{m}}{\text{s}})^2=0,\!579 \frac{\text{kN}}{\text{m}^2}
\end{equation}