\chapter{Beregning af $q_p$}
Nedenfor gives et beregningseksempel for beregning af $q_p$ for taget med vindretning fra vest. Fremgangsmåden for vindretningerne øst og nord er identisk med nedenstående eksempel, blot med andre værdier fra de samme kilder som anvendes i eksemplet, da vindretning og højden ændres.
\newline
\newline
Den maksimale belastning fra vinden, peakhastighedstrykket $q_p$, bestemmes ved:
\begin{center}
	$q_p(z_e)=[1+7I_v(z_e)]\frac{1}{2}pv_m^2(z_e)$
\end{center}
\begin{itemize}
	\item[-] $I_v$: vindturbulens
	\item[-] $\rho$: densiteten for luft ved $20^{\circ}$, $1,\!25 \frac{kg}{m^3}$
	\item[-] $v_m$: middelvindhastigheden
\end{itemize}
For at bestemme peakhastigheden, beregnes først vindturbulens $I_v(z)$ samt middelvindhastigheden $v_m$.
\newline
\newline
Vindturbulens, $I_v(z)$, bestemmes ved:
\begin{center}
	$I_v(z)=\frac{\sigma_v}{V_m(z)}=\frac{k_1}{c_0(z)\cdot ln(\frac{z}{z_0})}$
\end{center}
\begin{itemize}
	\item[-] $k_1$: turbulensfaktor, sættes til $1,\!0$ \citep[ kapitel 4.4]{EU91}
	\item[-] $c_0(z)$: orografifaktoren, som sættes til $1,\!0$ \citep[ kapitel 4.3.1]{EU91}
	\item[-] $z$: højden, som med taget er 19 m
	\item[-] $z_0$: ruhedslængde, som sættes til $1,\!0$ for terrænkategori IV \citep[ tabel 4.1 kapitel 4.3.2]{EU91}
\end{itemize}
Vindturbulensen bestemmes til:
\begin{center}
	$I_v(z)=\frac{1,\!0}{1,\!0\cdot ln(\frac{19}{1,\!0})}=0,\!340$
\end{center}
Middelvindhastigheden, $v_m$, bestemmes ved:
\begin{center}
	$v_m(z)=c_r(z)c_0(z)v_b$
\end{center}
\begin{itemize}
	\item[-] $c_r(z)$: ruhedsfaktor
	\item[-] $v_b$: basisvindhastigheden
\end{itemize}
Til at bestemme middelvindhastigheden, beregnes basisvindhastigheden samt ruhedsfaktoren.
\newline
\newline
Basisvindhastigheden, $v_b$, bestemmes ved:
\begin{center}
	$v_b=c_{dir}c_{season}v_{b,0}$
\end{center}
\begin{itemize}
	\item[-] $c_{dir}$: retningsfaktor, som sættes til $1,\!0$ ved vind fra vest \citep[ tabel 1a kapitel 4.2]{EU91}
	\item[-] $c_{season}$: årstidsfaktor, som sættes til $1,\!0$ \citep[ tabel 1b kapitel 4.2]{EU91}
	\item[-] $v_{b,0}$: grundværdi for basisvindhastigheden, som sættes til 24 $\frac{m}{s}$ \citep[ kapitel 4.2]{EU91}
\end{itemize}
Basisvindhastigheden bestemmes til:
\begin{center}
	$v_b=1,\!0\cdot 1,\!0\cdot 24 \frac{m}{s}=24 \frac{m}{s}$
\end{center}
Ruhedsfaktor, $c_r(z)$, bestemmes ved:
\begin{center}
	$c_r(z)=k_rln(\frac{z}{z_0})$
\end{center}
\begin{itemize}
	\item[-] $k_r$: terrænfaktor
\end{itemize}
Terrænfaktoren, $k_r$, bestemmes ved:
\begin{center}
	$k_r=0,\!19\cdot (\frac{z_0}{z_{0,II}})^{0,\!07}$
\end{center}
\begin{itemize}
	\item[-] $z_{0,II}$: værdi for ruhedslængde for terrænkategori II, som sættes til $0,\!05$ \citep[ kapitel 4.3.2]{EU91}
\end{itemize}
\begin{center}
	$k_r=0.19\cdot (\frac{1,0}{0,05})^{0,07}=0,\!234$
\end{center}
Ruhedsfaktor bestemmes til:
\begin{center}
	$c_r(z)=0,\!234\cdot ln(\frac{19}{1.0})=0,\!690$
\end{center}
Middelvindhastigheden bestemmes til:
\begin{center}
	$v_m(z)=0,\!690\cdot 1,\!0\cdot 24 \frac{m}{s}=16,\!569 \frac{m}{s}$
\end{center}
Peakhastighedstrykket $q_p$ i højden z, bestemmes til:
\begin{center}
	$q_p(z_e)=[1+7\cdot 0,\!340]\cdot \frac{1}{2}\cdot 1,\!25 \frac{kg}{m^3}\cdot (16,\!569 \frac{m}{s})^2=0,\!579 \frac{kN}{m^2}$
\end{center}