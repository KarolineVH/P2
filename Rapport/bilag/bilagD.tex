\chapter{Forsøg: Løs og fast lejring}

\textbf{Formål}
\newline
Formålet med dette forsøg er at bestemme hvor meget materiale der er tilbageholdt, som bruges til at finde lejringstætheden og poretallet af materialet.
\newline
\newline
\textbf{Apperaturliste}
\begin{itemize}
	\item[-] Lille cylinder
	\item[-] Tragt
	\item[-] Stållineal
	\item[-] Stamper passende til valgte cylinder
	\item[-] Specialskydelære passende til valgte cylinder
	\item[-] Vægt, vejenøjagtighed $0,\!01$ g
	\item[-] Sigte med maskevidde på 5 mm 
\end{itemize}

\textbf{Fremgangsmåde}
\newline
\underline{Løs lejring}
\newline
En delprøve af jordprøven er taget fra til forsøgende. Der startes med at placere cylinderen i en bakke. I cylinderen anbringes sigten. Så hældes materialet forsigtigt  op i sigten. Det skal glide ned ad kanten på sigten og videre i cylinderen. Der bruges tilstrækkeligt materiale at cylinderen kan fyldes helt op og så noget går over kanten. Sigten hæves nu forsigtigt op ad cylinderen. Det skal gøres over ca. et minut  i en jævn og flydende bevægelse. Når alt materialet er blevet hældt i cylinderen fjernes toppen med en stållineal således at materialet flågter med cylinderens overflade. Ved siden af cylinderet slås der hårdt to gange i bordet, så materialet sætter sig. Materiale der skulle befinder sig på ydersiden af cylideren børstes væk. Herefter vejes cylinderen og materialet, \textit{Cyl+Ws}. Cylinderen tømmes for materiale , børstes og vejes \textit{Cyl}.
\newline
\newline
\underline{Fast lejring}
\newline
Samme delprøve fra løs lejring bruges til fast lejring. Til fast lejring bliver cylinderet fyldt ad fem gange. Mellem hver fyldning, bliver materialet jævnet ud med en stållineal derefter stampes der. Antallet af slag stiger for hvert lag der kommer i. Stamperen holdes lodret i cylinderen. Faldloddet føres op til stopklodsen for at slippes og foretage frit fald. For hvert tiende slag tages stamperen for at der ikke fastklemmes materiae mellem stamperen og cylinderetsvæg. Stamperens fod børstes for materiale som eventuelt skulle være presset mod. Inden det sidste lag skal stampes skal der være ca. 0.5 cm fra materialet og op til toppen af cylinderet. Efter sidste stampning fjernes materiale der skulle sidde fast på stamperen og skydelæren sættes på kanten af cylinderet. Højden ned til materialet måles. Materiale som skulle være på ydersiden af cylinder børstes væk og cylinderen med materialet vejes \textit{Cyl+Ws}. Materialet fjernes og cylinderen bliver børstes og vejet \textit{Cyl}.
\newline
\newline
\textbf{Resultater for løs lejring}
\begin{center}
	\begin{tabular}{ |c|c|c| } 
		\hline
		Prøve nr & 1 & 2 \\	\hline 
		Areal [$cm^2$] & 10,0 & 10,0 \\ \hline
		Højde [cm] & 7,0 & 7,0 \\ \hline
		Volume [$cm^3$] & 70 & 70 \\ \hline
		Cyl + $W_s$ [g] & 340,45 & 340,46 \\ \hline
		Cyl [g] & 241,99 & 241,99 \\ \hline
		$W_s$ [g] & 98,55 & 98,47 \\ \hline
		e & 0,873 & 0,875 \\ \hline
	\end{tabular}
\end{center}

\begin{center}
	\begin{tabular}{ |c|c|c| } 
		\hline
		Prøve nr & 3 & 4 \\	\hline
		Areal [$cm^2$] & 10,0 & 10,0 \\ \hline
		Højde [cm] & 7,0 & 7,0 \\ \hline
		Volume [$cm^3$] & 70 & 70 \\ \hline
		Cyl + $W_s$ [g] & 340,50 & 337,52 \\ \hline
		Cyl [g] & 241,99 & 241,99 \\ \hline
		$W_s$ [g] & 98,51 & 98,57 \\ \hline
		e & 0,874 & 0,873\\ \hline
	\end{tabular}
\end{center}

\textbf{Resultater for fast lejring}
\begin{center}
	\begin{tabular}{ |c|c|c| } 
		\hline
		Prøve nr & 1 & 2 \\	\hline 
		Areal [$cm^2$] & 10,0 & 10,0 \\ \hline
		Højde [cm] & 6,5 & 6,575 \\ \hline
		Volume [$cm^3$] & 65 & 65,75 \\ \hline
		Cyl + $W_s$ [g] & 349,48 & 349,31 \\ \hline
		Cyl [g] & 241,99 & 238,95 \\ \hline
		$W_s$ [g] & 107,49 & 110,36 \\ \hline
		e & 0,595 & 0,571 \\ \hline
	\end{tabular}
\end{center}

\begin{center}
	\begin{tabular}{ |c|c|c| } 
		\hline
		Prøve nr & 3 & 4 \\	\hline
		Areal [$cm^2$] & 10,0 & 10,0 \\ \hline
		Højde [cm] & 6,42 & 6,355 \\ \hline
		Volume [$cm^3$] & 64,2 & 63,55 \\ \hline
		Cyl + $W_s$ [g] & 346,55 & 345,75 \\ \hline
		Cyl [g] & 238,95 & 238,95 \\ \hline
		$W_s$ [g] & 107,6 & 106,8 \\ \hline
		e & 0,573 & 0,569 \\ \hline
	\end{tabular}
\end{center}

\textbf{Fejlkilder}
\newline
I forsøggentagelse 2 i fast lejring blev der fyldt for meget materiale i cylinderen. Derfor er der taget noget ud efter de 80 slag og herefter er der givet 10 ekstra slag, hvilket kan give anledning til et varierende resultat. Det ses også, at resultat $W_s$ varierer lidt i forhold til de resterende forsøg.