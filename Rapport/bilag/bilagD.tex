\chapter{Forsøg: Løs og fast lejring}

\textbf{Formål}
\newline
Formålet med dette forsøg er, at bestemme hvor meget materiale der er tilbageholdt, som bruges til at finde lejringstætheden og poretallet af materialet.
\newline
\newline
\textbf{Apperaturliste}
\begin{itemize}
	\item[-] Lille cylinder
	\item[-] Tragt
	\item[-] Stållineal
	\item[-] Stamper passende til valgte cylinder
	\item[-] Specialskydelære passende til valgte cylinder
	\item[-] Vægt med vejenøjagtighed $0,\!01$ g
	\item[-] Sigte med maskevidde på 5 mm 
\end{itemize}

\textbf{Fremgangsmåde}
\newline
\underline{Løs lejring}
\newline
En delprøve af jordprøven er taget fra til forsøgene. Cylinderen placeres i bakken, og sigten anbringes i cylinderen. Materialet hældes forsigtigt op i sigten. Det skal glide ned ad kanten på sigten og videre i cylinderen. Der bruges tilstrækkeligt materiale til at cylinderen kan fyldes helt op. Sigten hæves nu forsigtigt op ad cylinderen. Det skal gøres over ca. et minut  i en jævn og flydende bevægelse. Når alt materialet er hældt i cylinderen fjernes toppen med en stållineal således at materialet flugter med cylinderens overflade. Ved siden af cylinderen slåes der hårdt to gange i bordet, så materialet sætter sig. Materialet der eventuelt befinder sig på ydersiden af cylinderen børstes væk. Herefter vejes cylinderen og materialet, \textit{Cyl+Ws}. Cylinderen tømmes for materiale, børstes og vejes \textit{Cyl}. Dette gentages 3-4 gange.
\newline
\newline
\underline{Fast lejring}
\newline
Samme delprøve fra løs lejring bruges til fast lejring. Til fast lejring fyldes cylinderen op fordelt over fem gange. Mellem hver fyldning bliver materialet jævnet ud med en stållineal, og derefter stampes der. Antallet af slag stiger for hvert lag, der kommes i. Stamperen holdes lodret i cylinderen. Faldloddet føres op til stopklodsen for at slippes og foretage frit fald. For hvert 10. slag tages stamperen op, og børstes ren, for at sikre, at der ikke fastklemmes materiae mellem stamperen og cylinderens væg. Stamperens fod børstes for materiale som eventuelt skulle være presset mod. Inden det sidste lag stampes, skal der være ca. $5,\!0$ mm fra materialet og til toppen af cylinderen. Efter sidste stampning fjernes det materiale, der eventuelt sidder fast på stamperen, og skydelæren sættes på kanten af cylinderen. Højden ned til materialet måles. Materialet, der eventuelt er på ydersiden af cylinderen, børstes væk, og cylinderen med materialet vejes og noteres \textit{Cyl+Ws}. Materialet fjernes og cylinderen børstes og vejes \textit{Cyl}.
\newline
\newline
\textbf{Resultater for løs lejring}
\begin{table}
\begin{center}
	\begin{tabular}{ |c|c|c|c|c| } 
		\hline
		Prøve nr. & 1 & 2 & 3 & 4 \\	\hline 
		Areal [$cm^2$] & 10,0 & 10,0 & 10,0 & 10,0 \\ \hline
		Højde [cm] & 7,0 & 7,0 & 7,0 & 7,0 \\ \hline
		Volume [$cm^3$] & 70 & 70 & 70 & 70 \\ \hline
		Cyl + $W_s$ [g] & 340,45 & 340,46 & 340,50 & 337,52 \\ \hline
		Cyl [g] & 241,99 & 241,99 & 241,99 & 241,99 \\ \hline
		$W_s$ [g] & 98,55 & 98,47 & 98,51 & 98,57 \\ \hline
		$e_{max}$ & 0,873 & 0,875 & 0,874 & 0,873 \\ \hline
	\end{tabular}
	\caption{Resultater for løs lejring}
	\label{tab:bilagd1}
\end{center}
\end{table}

\textbf{Resultater for fast lejring}
\begin{table}
\begin{center}
	\begin{tabular}{ |c|c|c|c|c| } 
		\hline
		Prøve nr. & 1 & 2 & 3 & 4 \\	\hline 
		Areal [$cm^2$] & 10,0 & 10,0 & 10,0 & 10,0 \\ \hline
		Højde [cm] & 6,500 & 6,575 & 6,420 & 6,355 \\ \hline
		Volume [$cm^3$] & 65,00 & 65,75 & 64,20 & 63,55 \\ \hline
		Cyl + $W_s$ [g] & 349,48 & 349,31 & 346,55 & 345,75 \\ \hline
		Cyl [g] & 241,99 & 238,95 & 238,95 & 238,95 \\ \hline
		$W_s$ [g] & 107,49 & 110,36 & 107,60 & 106,80 \\ \hline
		$e_{min}$ & 0,595 & 0,571 & 0,573 & 0,569 \\ \hline
	\end{tabular}
	\caption{Resultater for fast lejring}
	\label{tab:bilagd2}
\end{center}
\end{table}

\textbf{Fejlkilder}
\newline
I forsøg nr. 2 for fast lejring blev der fyldt for meget materiale i cylinderen. Derfor er der taget noget ud efter de 80 slag, og herefter er der givet 10 ekstra slag, hvilket kan give anledning til et varierende resultat. Det ses også, at resultat $W_s$ varierer lidt i forhold til de resterende forsøg.
\newline \indent{     }  Ved fast lejring forsøg 2 og 3 blev stampen ikke rengjort mellem hvert 10. slag. Dette kan have mindsket effekten af slagene, da kornene i cylinderen kan have givet friktion.