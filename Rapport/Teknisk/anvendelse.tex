\chapter{Anvendelsesgrænsetilstand}
Nu er brudgrænsetilstanden bestemt, og derfor skal der nu regnes for anvendelsesgrænsetilstanden. Hertil er der udregnet nye reaktioner for konstruktionen, da vindlasten optræder som eneste variable last. Vindlasterne på siden bliver forlænget, så denne virker ned i kælderen også, og der ses nu bort fra jordlasten, og der tages derfor fem snit, som vist på Figur 2. Ud fra anvendelsesgrænsetilstanden kan der vurderes, om bygningen opnår tilstrækkeligt små deformationer til, at bygningens dimensioner kan godtages, og denne kan konstrueres. Hertil skal bjælkens differentialligning anvendes, for at udregne udbøjningerne for stængerne. Fordi det statiske system for tilbygningen til Strøybergs Palæ er statisk bestemt, så er det den anden ordens afledede, som skal anvendes for at bestemme udbøjningerne. 
\newline
\newline
Nedenfor er vist et eksempel på, hvordan snitkræfterne regnes for snit 1. Alle beregningerne findes i BILAG XX.
\newline
\newline
Her ses en tabel med resultaterne for alle snitkræfterne. Ud fra denne skal ...