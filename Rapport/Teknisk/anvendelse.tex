\chapter{Anvendelsesgrænsetilstand}
Nu er brudgrænsetilstanden bestemt, og derfor kan der regnes for anvendelsesgrænsetilstanden. Hertil er der udregnet nye reaktioner for konstruktionen, da vindlasten optræder som eneste variable last. Vindlasterne på siden bliver forlænget, så denne virker ned i kælderen også, og der ses nu bort fra jordlasten. Derfor fås fem snit, som vist på Figur 2. Ud fra anvendelsesgrænsetilstanden kan der vurderes, om bygningen har tilstrækkelig små deformationer til, at bygningens dimensioner kan godtages. Hertil anvendes bjælkens differentialligning, for at udregne udbøjningerne for stængerne. Da det statiske system for tilbygningen til Strøybergs Palæ er statisk bestemt, er det den anden ordens afledede, som anvendes for at bestemme udbøjningerne. Reaktionerne for anvendelsesgrænsetilstanden udregnes på samme måde som reaktionerne for brudgrænsetilstand, der kan ses i Bilag ZZ. Værdierne ses i Tabel \ref{tab:anden}.
\begin{table}
	\begin{center}
		\begin{tabular}{|c|c|c|}
			\hline
			Reaktion & Værdi & Enhed \\ \hline
			$S_x$ & -71,14 		& kN      \\ \hline
			$C_y$ & 578,86 		& kN      \\ \hline
			$S_y$ & 36,07 		& kN       \\ \hline
			$B_y$ & 1432,31 	& kN      \\ \hline
			$A_y$ & 344,53 		& kN      \\ \hline
			$B_x$ & 0,00 		& kN      \\ \hline
			$A_x$ & -147,87 	& kN       \\ \hline
		\end{tabular}
		\caption{Reaktioner for anvendelsesgrænsetilstand}
		\label{tab:anden}
	\end{center}
\end{table}

Nedenfor er vist et eksempel på, hvordan snitkræfterne regnes for snit 1. Alle beregningerne findes i Bilag XX og resultaterne kan ses i Tabel \ref{tab:anden2}.
\begin{table}
	\begin{center}
		\begin{tabular}{|c|c|c|c|c|c|c|}
			\hline
			Snit/Værdi    & N - 0 m & N - $X_{max}$ m & V - 0 m & V - $X_{max}$ m & M - 0 m & M - $X_{max}$ m 	\\ \hline
			1. A-D, 0<x<18 	& -886,36 	& -318,62 	&  147,87 	&  38,39 	&  0,00     &  1676,36        		\\ \hline
			2. D-E, 0<x<6  	&  71,14    &  71,14    &  275,67   & -283,12   &  1676,36  &  $-9,00\cdot10^-6$    \\ \hline
			3. B-E, 0<x<18  & -948,58   &  164,54   &  0,00     &  0,00     &  0,00     &  0,00 			    \\ \hline
			4. E-F, 0<x<6   &  71,00    &  71,00    &  28,62    &  36,07    &  194,06   &  $1,00\cdot10^-6$     \\ \hline
			5. C-F, 0<x<18     & -578,86   & -11,12    &  0,00     &  21,56    &  0,00     &  194,06       		\\ \hline
		\end{tabular}
		\caption{Snitkræfter for anvendelsesgrænsetilstand, N og V: kN og M: $kN\cdot m$}
		\label{tab:anden2}
	\end{center}
\end{table}
