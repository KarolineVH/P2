\textbf{Egenlast af stålsystemet}
\newline
Egenlast stål i tagsystemet:
\newline

Densiteten for den anvendte ståltype S235 med profil nr. 450 har densiteten $g=115\frac{kg}{m}$ \citep{Stabi}. Dette skal ganges med højden af sadeltaget, som er $6,\!7m$. Dette skal også ganges med to, da der går to stålstænger op til taget. Derefter omregnes der til Newton
\begin{center}
	$115\frac{kg}{m}\cdot 6,\!7m \cdot 2 = 1542,\! 84 kg$
\end{center}

\begin{center}
	$1542,\! 84kg\cdot 9,\! 82\frac{m}{s^2} = 15,\! 150 kN$
\end{center}


Stålsystem eksklusiv tag:
\newline
Stålstyrken antages igen at være S235, som har densiteten $g=115\frac{kg}{m}$ \citep{Stabi}. For at finde kraften af en enkelt lodret stang multipliceres denne med højden af bygningen, som er 18 m, inklusiv kælderen.
\begin{center}
	$115\frac{kg}{m}\cdot 18m = 2070 kg$
\end{center}

\begin{center}
	$2070 kg \cdot 9,\!82\frac{m}{s^2} = 20,\! 33 kN$
\end{center}

Dette skal omregnes til at blive en linjelast og der divideres derfor med højden igen:

\begin{center}
$\frac{20,\! 33 kN}{18m} = 1,\! 13\frac{kN}{m}$
\end{center}

Det samme skal gøres for at finde kraften for en vandret stang, og densiteten skal her ganges med længden af den vandrette bjælke, som er 6 m.
\begin{center}
	$115\frac{kg}{m}\cdot 6m = 690 kg$
\end{center}

\begin{center}
	$690 kg \cdot 9,\!82\frac{m}{s^2} = 6,\!78  kN$
\end{center}

Dette skal omregnes til at blive en linjelast og der divideres derfor med længden af den vandrette stang:
$\frac{6,\! 78 kN}{6m} = 1,\! 13\frac{kN}{m}$
