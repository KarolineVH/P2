\chapter{Dimensionering}

Figur \ref{fig:hej} viser de nye byggefelter inden for henholdsvis delområde A og delområde B til Strøybergs Palæ (\citep{lokalplan}, s. 16). Denne rapport fokuserer på byggefeltet inden for delområde B, hvor ny bebyggelse, ifølge lokalplan 1-1-107, må opføres i 3 etager samt en tagetage og med en kælder maksimalt 2 m over terræn. Ved opførsel af ny bebyggelse i delområde B, skal to nuværende mindre bygninger fjernes. 

\begin{figure}[htbp]
	\centering
	\includegraphics[width=0.7\textwidth]{billeder/signatur.png}
	\caption{Lokalplan 1-1-107, delområde A og B \citep{lokalplan[ bilag 2, s. 35]}}
	\label{fig:hej}
\end{figure}

Med udgangspunkt i lokalplan 1-1-107 har bygningen fået de størrelser og dimensioner, som ses på Figur \ref{fig:farvel}.
\newline \indent{     }  Tilbygningen bliver 12,5 meter lang og 12 meter bred i henhold til den eksisterende bygningsbredde. Kælderen har en højde på i alt 3,25 m, hvor 1,25 m ligger over terræn. Stueetagen, 1. sal og 2. sal har hver især en højde på 4,9 m og tagetagen har en højde på 3 meter med en hældning på 26,6 grader. I alt er tilbygningen 19 m høj over terræn.

\begin{figure}[htbp]
	\centering
	\includegraphics[width=0.7\textwidth]{billeder/tilbygning2.png}
	\caption{Tilbygningens dimensioner}
	\label{fig:farvel}
\end{figure}

For at kunne beregne de laster som påvirker tilbygningen, er der opstillet nedenstående statisk system for bygningen. Systemet er opstillet som en bjælkekonstruktion.
\newline
\newline
Beregningerne opdeles for to gavle og to facader. Det antages at gavlene hver er 12 m lang og 16 m høj, eksklusiv en gavltrekant med en højde på 3 m, og har 9 vinduer med dimensionerne; 0,95 m bred og 1,5 m høj.
\newline
\newline
De to facader er 12,5 m lang og 16 m høj. Den har elleve vinduer, med samme mål som for vinduerne på gavlene, og en dør, som er 1,5 m bred og 2,1 m høj.
\newline
\newline
For at kunne beregne de laster, som påvirker tilbygningen, er der opstillet et statisk system for tilbygningen. Systemet er opstillet som en bjælkekonstruktion og indeholder tre rammekonstruktioner som vist på Figur \ref{fig:system}.

\begin{figure}[htbp]
	\centering
	\includegraphics[width=0.3\textwidth]{billeder/del1statiskesystem.png}
	\caption{Statiske system}
	\label{fig:system}
\end{figure}

På disse rammekonstruktioner vil etagedækkene virke som en belastning, i stedet for at virke som en del af konstruktion. Dette er muligt, da der kan opsættes en samling mellem etagedækkene og stålkonstruktion, som ses på Figur \ref{fig:etage}.

\begin{figure}[htbp]
	\centering
	\includegraphics[width=0.7\textwidth]{billeder/etageovergang.png}
	\caption{Etageovergang på tilbygningen}
	\label{fig:etage}
\end{figure}

Med denne opstilling kan reaktionerne beregnes og der ud fra finde belastningen, som vil komme på funderingen, men først skal lasterne beregnes.

\section{Laster}
Lidt tekst her.

\subsection{Permanent last}
Indsætte her.

\subsection{Variable laster}
Af variable laster optræder der både snelast, vindlast og nyttelast på bygningen, og disse udregnes efter Dansk Standard Eurocode 1991.

\subsubsection{Snelast}
Til at beregne hvordan snelasten påvirker tilbygningen anvendes den karaktiske snelast og formlen:
\begin{center}
$s=\mu_i\cdot C_e\cdot C_t \cdot s_k$
\end{center}
\begin{itemize}
	\item[-] $s$: karakteristisk snelast
	\item[-] $\mu_i$: formfaktoren for snelasten, som sættes til 0.8 \citep[ tabel 5.2 kapitel 5.3]{EU91}
	\item[-] $C_e$: eksponeringsfaktoren
	\item[-] $C_t$: termisk faktor, som sættes til 1.0 \citep[ kapitel 5.2]{EU91}
	\item[-] $s_k$: karakteristisk terrænværdi, som sættes til $1 \frac{kN}{m^2}$ \citep[ kapitel 4.1]{EU91}
\end{itemize}
Til at bestemme den karakteristiske snelast, beregnes eksponeringsfaktoren $C_e$.
\newline
\newline
Eksponeringsfaktoren, $C_e$, bestemmes ved:
\begin{center}
$C_e=C_{top}\cdot C_s$
\end{center}
\begin{itemize}
	\item[-] $C_{top}$: topografi faktor, som sættes til 1.0 \citep[ tabel 5.1 kapitel 5.2]{EU91}
	\item[-] $C_s$: størrelse faktor, som sættes til 1.0 \citep[ kapitel 5.2]{EU91}
\end{itemize}
Eksponeringsfaktoren kan nu bestemmes til:
\begin{center}
$C_e=1.0\cdot 1.0=1.0$
\end{center}
Strøybergs Palæ har et saddeltag, og dermed skal der tages højde for fire lasttilfælde, som ses på Figur INDSÆTTE FIGUR!
\newline
\newline
\underline{Lasttilfælde 1}
\begin{center}
$s_1=0.8\cdot 1.0\cdot 1.0\cdot 1 \frac{kN}{m^2}=0.8 \frac{kN}{m^2}$
\end{center}
\underline{Lasttilfælde 2 og 3}
\begin{center}
$s_2=\frac{1}{2}\cdot 0.8\cdot 1.0\cdot 1.0\cdot 1 \frac{kN}{m^2}=0.4 \frac{kN}{m^2}$
\end{center}
\underline{Lasttilfælde 4}
\begin{center}
	$s_4=\mu_w\cdot C_e\cdot C_t\cdot s_k \frac{kN}{m^2}$
\end{center}
\begin{itemize}
	\item[-] $\mu_w$: formfaktoren, som sættes til 1.2 eftersom $\alpha$ er $26.565^{\circ}$ \citep[ kapitel 5.3.3]{EU91}
\end{itemize}
Den karakteristiske snelast for lasttilfælde 4 kan nu bestemmes til:
\begin{center}
	$s_4=1.2\cdot 1.0\cdot 1.0\cdot 1 \frac{kN}{m^2}=1.2 \frac{kN}{m^2}$
\end{center}
I og med at lasttilfælde 4 giver den største last, anvendes denne til videre beregning.

\subsubsection{Vindlast}
Vindlasten beregnes for det højeste punkt på konstruktionen, hvilket er på tagspidsen for tilbygningen, da det er det punkt, hvor vinden er kraftigst.
\newline
\newline
FIGUR! - henvis til den med taget, som skal være i egenlasten
\newline
\newline
Til at bestemme vindlasten på tilbygningen bruges følgende formel:	
\begin{center} $w_e=q_p(z_e)$$\cdot$$c_{pe}$
\end{center}
\begin{itemize}
	\item[-] $q_p$: peakhastighedstrykket
	\item[-] $z_e$: referencehøjden for det udvendige vindtryk
	\item[-] $c_{pe}$: formfaktoren for det udvendige vindtryk
\end{itemize}
Den maksimale belastning fra vinden, peakhastighedstrykket $q_p$, bestemmes ved:
\begin{center}
$q_p(z_e)=[1+7I_v(z_e)]$$\cdot$$\frac{1}{2}$$\cdot$p$\cdot$$v_m^2(z_e)$
\end{center}
\begin{itemize}
	\item[-] $I_v$: vindturbulens
	\item[-] $\rho$: densiteten for luft $1.25 \frac{kg}{m^3}$ KILDE
	\item[-] $v_m$: middelvindhastigheden
\end{itemize}
For at bestemme peakhastigheden, beregnes først vindturbulens $I_v(z)$ samt middelvindhastigheden $v_m$.
\newline
\newline
Vindturbulens, $I_v(z)$, bestemmes ved:
\begin{center}
$I_v(z)=\frac{\sigma_v}{V_m(z)}=\frac{k_1}{c_0(z)\cdot ln(\frac{z}{z_0})}$
\end{center}
\begin{itemize}
	\item[-] $k_1$: turbulensfaktor, sættes til 1.0 \citep[ kapitel 4.4]{EU91}
	\item[-] $c_0(z)$: orografifaktoren, som sættes til 1.0 \citep[ kapitel 4.3.1]{EU91}
	\item[-] $z$: højde, som er 19 m
	\item[-] $z_0$: ruhedslængde, som sættes til 1.0 for terrænkategori IV \citep[ tabel 4.1 kapitel 4.3.2]{EU91}
\end{itemize}
Vindturbulensen kan nu bestemmes til:
\begin{center}
$I_v(z)=\frac{1.0}{1.0\cdot ln(\frac{19}{1.0})}=0.340$
\end{center}
Middelvindhastigheden, $v_m$, bestemmes ved:
\begin{center}
$v_m(z)=c_r(z)\cdot c_0(z)\cdot v_b$
\end{center}
\begin{itemize}
	\item[-] $c_r(z)$: ruhedsfaktor
	\item[-] $v_b$: basisvindhastigheden
\end{itemize}
Til at bestemme middelvindhastigheden, beregnes basisvindhastigheden samt ruhedsfaktor.
\newline
\newline
Basisvindhastigheden, $v_b$, bestemmes ved:
\begin{center}
$v_b=c_{dir}\cdot c_{season}\cdot v_{b,0}$
\end{center}
\begin{itemize}
	\item[-] $c_{dir}$: retningsfaktor, som sættes til 1.0 \citep[ tabel 1a kapitel 4.2]{EU91}
	\item[-] $c_{season}$: årstidsfaktor, som sættes til 1.0 \citep[ tabel 1b kapitel 4.2]{EU91}
	\item[-] $v_{b,0}$: grundværdi for basisvindhastigheden, som sættes til 24 $\frac{m}{s}$, da dette er gældende for størstedelen af Danmark \citep[ kapitel 4.2]{EU91}
\end{itemize}
Basisvindhastigheden kan nu bestemmes til:
\begin{center}
$v_b=1.0\cdot 1.0\cdot 24 \frac{m}{s}=24 \frac{m}{s}$
\end{center}
Ruhedsfaktor, $c_r(z)$, bestemmes ved:
\begin{center}
$c_r(z)=k_r\cdot ln(\frac{z}{z_0})$
\end{center}
\begin{itemize}
	\item[-] $k_r$: terrænfaktor
\end{itemize}
Terrænfaktoren, $k_r$, bestemmes ved:
\begin{center}
$k_r=0.19\cdot (\frac{z_0}{z_{0,II}})^{0.07}$
\end{center}
\begin{itemize}
	\item[-] $z_{0,II}$: værdi for ruhedslængde for terrænkategori II, som sættes til 0.05 \citep[ kapitel 4.3.2]{EU91}
\end{itemize}
\begin{center}
$k_r=0.19\cdot (\frac{1.0}{z_{0.05}})^{0.07}=0.234$
\end{center}
Ruhedsfaktor kan nu bestemmes til:
\begin{center}
$c_r(z)=0.234\cdot ln(\frac{19}{1.0})=0.690$
\end{center}
Middelvindhastigheden kan nu bestemmes til:
\begin{center}
$v_m(z)=0.690\cdot 1.0\cdot 24 \frac{m}{s}=16.569 \frac{m}{s}$
\end{center}
Peakhastighedstrykket $q_p$ i højden z, kan nu bestemmes til:
\begin{center}
$q_p(z_e)=[1+7\cdot 0.340]\cdot \frac{1}{2}\cdot 1.25 \frac{kg}{m^3}\cdot (16.569 \frac{m}{s})^2=0.579 \frac{kN}{m^2}$
\end{center}

HVAD GØR VI NU??

\begin{figure}[htbp]
	\centering
	\includegraphics[width=0.5\textwidth]{billeder/opdeling.png}
	\caption{Zoneinddeling af taget}
	\label{fig:tag}
\end{figure}

For alle zoner bestemmes $c_{pe,10}$. Der opstilles en lineær ligning med sammenhæng mellem $c_{pe,10}$ værdierne og graderne 15 og 30. Herefter indsættes taghældningen, $26.565^{\circ}$ i ligningen og værdien for $c_{pe,10}$ i den pågældende zone fås.
\newline
\newline
\underline{Zone F}
\newline
Ud fra \citep[ tabel 7.4a kapitel 7.2.5]{EU91} er de negative værdier for zone F: -0.9 og -0.5. Her ud fra fås ligningen, og $c_{pe,10,neg}$ bestemmes:
\begin{center}
	$f(\alpha)=0.0267\cdot \alpha - 1.3 \to c_{pe,10,neg}=-0.592$
\end{center}
De positive værdier for zone F er: 0.2 og 0.7. Her ud fra fås ligningen, og $c_{pe,10,pos}$ bestemmes:
\begin{center}
	$f(\alpha)=0.0333\cdot \alpha - 0.3 \to c_{pe,10,pos}=0.585$
\end{center}

\underline{Zone G}
\newline
De negative værdier for zone G er: -0.8 og -0.5. Her ud fra fås ligningen, og $c_{pe,10,neg}$ bestemmes:
\begin{center}
	$f(\alpha)=0.02\cdot \alpha - 1.1 \to c_{pe,10,neg}=-0.569$
\end{center}
De positive værdier for zone G er: 0.2 og 0.7. Her ud fra fås ligningen, og $c_{pe,10,pos}$ bestemmes:
\begin{center}
	$f(\alpha)=0.0333\cdot \alpha - 0.3 \to c_{pe,10,pos}=0.585$
\end{center}

\underline{Zone H}
\newline
De negative værdier for zone H er: -0.3 og -0.2. Her ud fra fås ligningen, og $c_{pe,10,neg}$ bestemmes:
\begin{center}
	$f(\alpha)=0.00667\cdot \alpha - 0.4 \to c_{pe,10,neg}=-0.223$
\end{center}
De positive værdier for zone H er: 0.2 og 0.4. Her ud fra fås ligningen, og $c_{pe,10,pos}$ bestemmes:
\begin{center}
	$f(\alpha)=0.0133\cdot \alpha - 1.178\cdot 10^{-16} \to c_{pe,10,pos}=0.354$
\end{center}

\underline{Zone I}
\newline
Den negative værdi for zone I er: -0.4. Her ud fra fås ligningen, og $c_{pe,10,neg}$ bestemmes:
\begin{center}
	$f(\alpha)=-5.234\cdot 10^{-18}\cdot \alpha - 0.4 \to c_{pe,10,neg}=-0.4$
\end{center}
Den positive værdi for zone I er: 0.0. Her ud fra fås ligningen, og $c_{pe,10,pos}$ bestemmes:
\begin{center}
	$f(\alpha)=0.0 \to c_{pe,10,pos}=0.0$
\end{center}

\underline{Zone J}
\newline
De negative værdier for zone J er: -1.0 og -0.5. Her ud fra fås ligningen, og $c_{pe,10,neg}$ bestemmes:
\begin{center}
	$f(\alpha)=0.0333\cdot \alpha - 1.5 \to c_{pe,10,neg}=-0.615$
\end{center}
Den positive værdi for zone J er: 0.0. Her ud fra fås ligningen, og $c_{pe,10,pos}$ bestemmes:
\begin{center}
	$f(\alpha)=0.0 \to c_{pe,10,pos}=0.0$
\end{center}
INDSÆTTE DET SIDSTE!

\subsection{Nyttelast}
Ud fra \citep[ tabel 6.2 kapitel 6.3.1.2]{EU91} aflæses den jævnt fordelte last, $q_k$, for kategori A1, som er bolig og lokale adgangsveje, til at være $1,5 \frac{kN}{m^2}$. Denne last beregnes for alle etager på tilbygningen samlet set.
\begin{center}
	$Q_K3=q_k\cdot A\cdot antal etager$
\end{center}
\begin{itemize}
	\item[-] A: arealet af én etage, som sættes til $150,0 m^2$
	\item[-] antal etager for tilbygningen er 5, men der ses bort fra kælderetagen
\end{itemize}
Nyttelasten kan nu bestemmes til:
\begin{center}
	$Q_K3=1,5 \frac{kN}{m^2}\cdot 150,0 m^2\cdot 4=900,0 kN$
\end{center}

\section{Lastkombinationer}
INTROTEKST
\newline
\newline
\underline{Lastkombination 1: Egenlast}
\newline
Den regningsmæssige egenlasten bestemmes ud fra følgende formel:
\begin{center}
	$E_{d,1}=\gamma_{G1}\cdot K_{FI}\cdot G_{K1}$
\end{center}
\begin{itemize}
	\item[-] $\gamma_{G1}$: partialkoefficienten for den permanente last, som sættes til 1,2 \citep[ tabel A 1.2(B+C) anneks A.1.3.1]{EU90}
	\item[-] $K_{FI}$: konsekvensklasse CC3, som sættes til 1,1 \citep[ tabel A 1.2(A) anneks A.1.3.1]{EU90}
	\item[-] $G_{K1}$: karakteristisk egenlast for den permanente last [kN], som sættes til 3437,86 kN
\end{itemize}
Den regningsmæssige egenlast kan nu bestemmes til:
\begin{center}
	$E_{d,1}=1,2\cdot 1,1\cdot 3437,86 kN=4,538\cdot 10^3 kN$
\end{center
}
\underline{Lastkombination 2: Vindlast dominerende}
\newline
Den regningsmæssige last, for vindlast dominerende, bestemmes ud fra følgende formel:
\begin{center}
	$E_{d,2}=\gamma_{G1}\cdot K_{FI}\cdot G_{K1}+\gamma_{Q1}\cdot K_{FI}\cdot Q_{K1}+\gamma_{Q2}\cdot \Psi_{0,2}\cdot K_{FI}\cdot Q_{K2}+\gamma_{Q3}\cdot \Psi_{0,3}\cdot K_{FI}\cdot Q_{K3}$
\end{center}
\begin{itemize}
	\item[-] $\gamma_{Q1}$: partialkoefficienten, som sættes til 1,5 \citep[ tabel A 1.2(B+C) anneks A.1.3.1]{EU90}
	\item[-] $Q_{K1}$: karakteristisk nyttelast for vind, som sættes til X
	\item[-] $\gamma_{Q2}$: partialkoefficenten, som sættes til 1,5 \citep[ tabel A 1.2(B+C) anneks A.1.3.1]{EU90}
	\item[-] $\Psi_{0,2}$: $\Psi$-faktor, som sættes til 0,0, ved vindlast dominerende \citep[ tabel A 1.1 anneks A.1.2.2]{EU90}
	\item[-] $Q_{K2}$: karakteristisk nyttelast for sne, som sættes til Y
	\item[-] $\gamma_{Q3}$: partialkoefficenten, som sættes til 1,5 \citep[ tabel A 1.2(B+C) anneks A.1.3.1]{EU90}
	\item[-] $\Psi_{0,3}$: $\Psi$-faktor, som sættes til 0,5 \citep[ tabel A 1.1 anneks A.1.2.2]{EU90}
	\item[-] $Q_{K3}$: karakteristisk nyttelast for bolig, som sættes til 900,0 kN
\end{itemize}
Den regningsmæssige værdi for vindlast dominerende kan nu bestemmes til:
\begin{center}
	$E_{d,2}=$
\end{center}

\underline{Lastkombination 3: Snelast dominerende}
\newline
Den regningsmæssige last, for snelast dominerende, bestemmes ud fra følgende formel:
\begin{center}
 	$E_{d,3}=\gamma_{G1}\cdot K_{FI}\cdot G_{K1}+\gamma_{Q1}\cdot \Psi_{0,1}\cdot K_{FI}\cdot Q_{K1}+\gamma_{Q2}\cdot K_{FI}\cdot Q_{K2}+\gamma_{Q3}\cdot \Psi_{0,3}\cdot K_{FI}\cdot Q_{K3}$
\end{center}
\begin{itemize}
	$\Psi_{0,1}$: $\Psi$-faktor, som sættes til 0,3 \citep[ tabel A 1.1 anneks A.1.2.2]{EU90}
\end{itemize}
Den regningsmæssige værdi for snelast dominerende kan nu bestemmes til:
\begin{center}
	$E_{d,3}=$
\end{center}

\underline{Lastkombination 4: Nyttelast dominerende}
\newline
Den regningsmæssige last, for nyttelast dominerende, bestemmes ud fra følgende formel:
\begin{center}
	$E_{d,4}=\gamma_{G1}\cdot K_{FI}\cdot G_{K1}+\gamma_{Q1}\cdot \Psi_{0,1}\cdot K_{FI}\cdot Q_{K1}+\gamma_{Q2}\cdot \Psi_{0,2}\cdot K_{FI}\cdot Q_{K2}+\gamma_{Q3}\cdot K_{FI}\cdot Q_{K3}$
\end{center}
\begin{itemize}
	\item[-] $\Psi_{0,2}$: $\Psi$-faktor, som sættes til 0,3 \citep[ tabel A 1.1 anneks A.1.2.2]{EU90}
\end{itemize}
Den regningsmæssige værdi for nyttelast dominerende kan nu bestemmes til:
\begin{itemize}
	$E_{d,4}=$
\end{itemize}