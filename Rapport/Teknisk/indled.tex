\chapter{Dimensionering af stålprofiler}

Figur \ref{fig:hej} viser de nye byggefelter inden for henholdsvis delområde A og delområde B til Strøybergs Palæ (\citep{lokalplan}, s. 16). Denne rapport fokuserer på byggefeltet inden for delområde B, hvor ny bebyggelse, ifølge lokalplan 1-1-107, må opføres i 3 etager samt en tagetage og med en kælder maksimalt 2 m over terræn. Ved opførsel af ny bebyggelse i delområde B, skal to nuværende mindre bygninger fjernes. 

\begin{figure}[htbp]
	\centering
	\includegraphics[width=0.8\textwidth]{billeder/signatur.png}
	\caption{Lokalplan 1-1-107, delområde A og B \citep{lokalplan[ bilag 2, s. 35]}}
	\label{fig:hej}
\end{figure}

Med udgangspunkt i lokalplan 1-1-107 har bygningen fået de størrelser og dimensioner, som ses på Figur \ref{fig:farvel}.
\newline \indent{     }  Tilbygningen bliver $12,\!5$ meter lang og 12 meter bred i henhold til den eksisterende bygningsbredde. Kælderen har en højde på i alt $3,\!25$ m, hvor $1,\!25$ m ligger over terræn. Stueetagen, 1. sal og 2. sal har hver især en højde på $4,\!9$ m og tagetagen har en højde på 3 meter med en hældning på $26,\!6$ grader. I alt er tilbygningen 19 m høj over terræn.

\begin{figure}[htbp]
	\centering
	\includegraphics[width=0.7\textwidth]{billeder/tilbygning2.png}
	\caption{Tilbygningens dimensioner}
	\label{fig:farvel}
\end{figure}

For at kunne beregne de laster som påvirker tilbygningen, er der opstillet nedenstående statisk system for bygningen. Systemet er opstillet som en bjælkekonstruktion. Som det ses på Figur \ref{fig:farvel} er der tre stålrammer; en i midten og en i hver gavlende. I de videre beregninger foretagers der beregninger for rammen i midten.
\newline
\newline
Beregningerne opdeles for to gavle og to facader. Det antages at gavlene hver er 12 m lang og 16 m høj, eksklusiv en gavltrekant med en højde på 3 m, og har 9 vinduer med dimensionerne; $0,\!95$ m bred og $1,\!5$ m høj.
\newline
\newline
De to facader er $12,\!5$ m lang og 16 m høj. Den har 11 vinduer, med samme mål som for vinduerne på gavlene, og en dør, som er $1,\!5$ m bred og $2,\!1$ m høj.
\newline
\newline
For at kunne beregne de laster, som påvirker tilbygningen, er der opstillet et statisk system for tilbygningen. Systemet er opstillet som en bjælkekonstruktion og indeholder tre rammekonstruktioner som vist på Figur \ref{fig:system}.

\begin{figure}[htbp]
	\centering
	\includegraphics[width=0.3\textwidth]{billeder/del1statiskesystem.png}
	\caption{Statiske system}
	\label{fig:system}
\end{figure}

På disse rammekonstruktioner vil etagedækkene virke som en belastning, i stedet for at virke som en del af konstruktion. Dette er muligt, da der kan opsættes en samling mellem etagedækkene og stålkonstruktion, som ses på Figur \ref{fig:etage}.

\begin{figure}[htbp]
	\centering
	\includegraphics[width=0.2\textwidth]{billeder/etageovergang.png}
	\caption{Etageovergang på tilbygningen}
	\label{fig:etage}
\end{figure}

Med denne opstilling kan reaktionerne beregnes og der ud fra finde belastningen, som vil komme på funderingen, men først skal lasterne beregnes.

\section{Laster}
Tilbygningen til Strøybergs Palæ vil blive udsat for en række forskellige laster, både permanente- og variable laster, hvilke vil blive beregnet i dette afsnit således, at der senere hen kan opstilles lastkombinationer, regnes reaktionskræfter, snitkræfter, spændinger samt brudgrænse- og anvendelsesgrænsetilstande.

\subsection{Permanente laster}
Egenlaster er permanente laster. For tilbygningen til Strøybergs Palæ beregnes disse laster gennem de gjorte antagelser samt de givne mål, som er illustreret på Figur \ref{fig:farvel}.
\newline
\newline
\textbf{Egenlast}
\newline
Det antages at taget, som benyttes, er et mellemtungt tag, som har værdien $600 \frac{N}{m^2}$ \citep{tag}, og det er et sadeltag med teglsten. Der er derfor fundet en middelværdi for lasten fra taget til systemet. 
\newline
\newline
Tagets areal bestemmes ud fra Figur \ref{fig:tagetage}:
\begin{center}
	$6,\!7 m\cdot 12,\!5 m \cdot 2=167,\!500 m^2$
\end{center}

Arealet ganges med værdien for mellemtungt tag:
\begin{center}
	$167,\!500 m^2\cdot 600 \frac{N}{m^2}=100,\!500 kN$
\end{center}

\underline{Last fra tagsystem}
\newline
\textit{Last fra væg}
\newline
Facaderne antages for at være ens samt indeholde lige mange vinduer og døre. Der antages at være 11 vinduer og én dør pr. facade \citep{gammellokalplan}, hvilket illustreres på Figur \ref{fig:facade}.

\begin{figure}[htbp]
	\centering
	\includegraphics[width=0.8\textwidth]{billeder/facadenord.png}
	\caption{Facaden på øst- og vestsiden}
	\label{fig:facade}
\end{figure}

\indent{     }  Ét vindue antages til være $0,\!95$ m bred og $1,\!5$ m høj. Døren antages at være $1,\!5$ m bred og $2,\!1$ m høj. Arealet for én facade bestemmes til:
\begin{center}
	$12,\!5 m\cdot 16,\!0 m - (1,\!5 m\cdot0,\!95 m\cdot11 + 1,\!5 m\cdot 2,\!1 m)=181,\!175 m^2$
\end{center}

Til beregning af væggens egenlast vides det, at der skal påregnes en indervæg og et isoleringslag, mens den ydervæggen ikke er en del af det statiske system og dermed ikke skal medregnes.
\newline \indent{     }  En typisk indervæg er typisk 130 mm tyk \citep{indervaeg}, og det antages, at denne værdi benyttes til tilbygningen af Strøybergs Palæ. Dermed kan værdien for indervæggen af facaden nu bestemmes ved at gange facadens areal uden viduer og dør med tykkelsen, hvorefter denne værdi ganges med densiteten for mursten \citep{murstendensitet}, hvilket er $1500,\!0 \frac{kg}{m^3}$:
\begin{center}
	$181,\!175 m^2\cdot 0,\!130 m\cdot 1500,\!0 \frac{kg}{m^3}\cdot 9,\!82 \frac{m}{s^2}=35329,\!125 kg$
\end{center}

Denne værdi omregnes til $N$ ved at gange med tyngdeaccelerationen:
\begin{center}
	$35329,\!125 kg\cdot 9,\!82 \frac{m}{s^2}=346,\!932 kN$
\end{center}

I og med facaderne er identiske ganges de $346,\!932 kN$ med 2:
\begin{center}
	$346,\!932 kN\cdot 2=693,\!864 kN$
\end{center}

Isoleringslaget regnes til at være 100 mm tyk \citep{isolering}, og det antages, at typen af isolering har densiteten $30 \frac{kg}{m^3}$ \citep{densitet}.
\newline \indent{     }  Facadens areal uden vinduer og dør multipliceres med tykkelsen, hvorefter denne værdi multipliceres med densiteten for isoleringen:
\begin{center}
	$181,\!175 m^2\cdot 0,\!100 m \cdot 30 \frac{kg}{m^3}\cdot 9,\!82 \frac{m}{s^2}=5,\!337 kN$
\end{center}

I og med facaderne er identiske ganges de $5,\!337 kN$ med 2:
\begin{center}
	$5,\!337 kN\cdot 2=10,\!675 kN$
\end{center}

Da den ene gavl ligger op ad den nuværende bygning, vil der ikke være vinduer på denne gavl, mens der antages at være 9 vinduer, identiske med dem for facaden, for gavlen, der ikke ligger op ad den nuværende bygning, som kan ses på Figur \ref{fig:gavl}.

\begin{figure}[htbp]
	\centering
	\includegraphics[width=0.6\textwidth]{billeder/facadevestellerost.png}
	\caption{Dimensioner for tagetagen}
	\label{fig:gavl}
\end{figure}

Til forskel for facadens areal, har gavlene også arealet for tagetagen, som har dimensionerne der ses på Figur \ref{fig:tagetage}. Arealet for facaden der ikke ligger op ad den nuværende bygning bestemmes dermed til:
\begin{center}
	$12 m\cdot 16 m + 3 m\cdot 12 m \cdot 0,\!5 - 1,\!5 m\cdot 0,\!95 m \cdot 9=197,\!175 m^2$
\end{center}

\begin{figure}[htbp]
	\centering
	\includegraphics[width=0.7\textwidth]{billeder/Tagmedvinkel.png}
	\caption{Dimensioner for tagetagen}
	\label{fig:tagetage}
\end{figure}

Ligesom for facaderne skal der også påregnes en indervæg og et isoleringslag for gavlen, og det antages at der anvendes den samme tykkelse og densitet for indervæg og isolering.
\newline
\newline
Værdien for indervæggen bestemmes til:
\begin{center}
	$197,\!175 m^2\cdot 0,\!130 m\cdot 1500,\!0 \frac{kg}{m^3}\cdot 9,\!82 \frac{m}{s^2}=377,\!570 kN$
\end{center}

Værdien for isolering bestemmes til:
\begin{center}
	$197,\!175 m^2\cdot 0,\!100 mcdot 30 \frac{kg}{m^3}\cdot 9,\!82 \frac{m}{s^2}=5,\!809 kN$
\end{center}

\textit{Last fra gulv}
\newline
Det antages at bygnings etager kun består af gulv, dvs. ingen skillevægge, trapper og andre former for genstande, der mindsker mængden af gulv. Derfor er den samlede mængde gulv pr. etage $12,\!5 m\cdot 12,\!0 m=150,\!0 m^2$.
\newline \indent{     }  Fire ud af de fem etager består af bærende gulv, hvor den sidste, kælderetagen, ligger på fundamentet. Et bærende gulv antages at bestå af et armeret betondæk nederst, der oftest er mellem 80 mm og 200 mm tyk. Der antages, at den er 120 mm tyk, hvorpå der ligger bjælker. Bjælkernes opgave er at lede og overføre lasterne ud i understøtning. Derefter vil ligge et undergulv, mellemgulv og til sidst selve gulvbelægningen, og mellem betondækket og undergulvet lægges isoleringen \citep{Gulvopbygning}. 
\newline \indent{     }  Først bestemmes gulvets last for en etage, hvorefter denne ganges denne med fire, da der er fire etager foruden kælderetagen. Betondækkets andel beregnes først, da både længden, bredden og tykkelsen kendes. 
\newline
\newline
Rumfanget af betonen for én etage bestemmes til:
\begin{center}
	$12,\!5 m\cdot 12,\!0 m\cdot 0,\!120 m=18,\!0 m^3$
\end{center}

Rumfanget ganges med betons densitet, som er $2400 \frac{kg}{m^3}$ \citep{betonsdensitet}, hvorefter hvorefter denne værdi ganges med tyngdeaccelerationen, for at få værdien i $kN$, som der ganges med 4 for at få lasten for alle fire etager:
\begin{center}
	$18,\!0 m^3\cdot 2400 \frac{kg}{m^3}\cdot 9,82 \frac{m}{s^2}\cdot 4=1693,440 kN$
\end{center}

Der er behov for bjælker, som hver er $6,\!25$ m lang, da de er halvdelen af tilbygningens længde, $140,\!0$ mm høj og $140,\!0$ mm bred \citep{granse}. Disse anlægges langs med tilbygningen. For at bestemme hvor mange bjælker, der er behov for, divideres længden af tilbygningen med 400 mm, da der skal være 400 mm imellem hver bjælke \citep{Gulvopbygning}. 

\begin{center}
	$\frac{12,\!5 m}{0,\!400 m}=31,\!25$ bjælker
\end{center} 

Da der ikke kan optræde $31,\!25$ bjælker rundes der op til 32 bjælker.
\newline
\newline
Rumfanget af én bjælke, som består af to bjælker med længden $6,\!25$ m, ganges med bjælkens densitet for tørrumvægt af træ \citep{torrumvagt}, hvorefter denne værdi ganges med tyngdeaccelerationen og ganges med 4: 
\begin{center}
	$7,\!840 m^3\cdot 510 \frac{kg}{m^3}\cdot 9,\!82 \frac{m}{s^2}\cdot 4=157,\!057 kN$
\end{center}

Isoleringsarealet for tilbygningen er det samlede gulvareal minus det areal, som bjælkerne ligger på:
\begin{center}
	$150,\! m^2 - (0,\!140 m\cdot 12,\!5 m\cdot 32)=94,\!0 m^2$
\end{center}

Isoleringslaget har samme højde som bjælkerne og har en densitet på $30 \frac{kg}{m^3}$ \citep{densitet}. Dermed kan værdien for isolering bestemmes til:
\begin{center}
	$94,\!0 m^2\cdot 0,\!140 m\cdot 30 \frac{kg}{m^3}\cdot 9,\!82 \frac{m}{s^2}\cdot 4=15,\!508 kN$
\end{center}

Gulvbelægningen antages for at være linoleumsgulv med en højde på $2,\!5 mm$ og en densitet på $2,\!9 \frac{kg}{m^3}$ \citep{linoleum}. 
Dermed kan værdien for linoleumsgulv bestemmes til:
\begin{center}
	$0,\!0025 m\cdot 12,\!5 m\cdot 12,\!0 m\cdot 2,\!9 \frac{kg}{m^3}\cdot 9,\!82 \frac{m}{s^2}\cdot 4=0,\!043 kN$
\end{center}

\textit{Samlet egenlast eksklusiv tag og stålsystem}
\newline
De tidligere beregnede værdier adderes; 
\begin{center}
	$346,\!932 kN + 346,\!932 kN + 5,\!337 kN + 5,\!337 kN + 377,\!509 kN + 5,\!809 kN + 1693,\!440 kN + 157,\!057 kN + 15,\!508 kN + 0,\!043 kN = 2953,\!904 kN$
\end{center}
Den samlede egenlast deles med 2, da midterrammen optager halvdelen af den beregnede last. 
\begin{center}
	$\frac{2953,904 kN}{2} =  1476,\!952 kN$
\end{center}
Den midterste stang optager $\frac{1}{2}$ af egenlasten, mens de to yderste stænger hver optager $\frac{1}{4}$, derfor divideres nu med 4.

\begin{center}
	$\frac{1476,952 kN}{4} =  369,\!24 kN$
\end{center}

Denne last laves nu om til en linjelast ned langs stængerne, og derfor deles med 18m:
 
\begin{center}
	$\frac{369,24 kN}{18m} =  20,\!51 \frac{kN}{m}$
\end{center} 

\textit{Egenlast af stålsystemet}
\newline
Egenlast stål i tagsystemet:
\newline
Densiteten for den anvendte ståltype S235 med profil nr. 450 har densiteten $g=115\frac{kg}{m}$ \citep{Stabi}. Dette skal ganges med højden af sadeltaget, som er $6,\!7m$ og omregnes efterfølgende til KiloNewton.  
\begin{center}
	$115\frac{kg}{m}\cdot 6,\!7m = 771,\! 42 kg$
\end{center}

\begin{center}
	$771,\! 42kg\cdot 9,\! 82\frac{m}{s^2} = 7,\! 5753444 kN$
\end{center}

Denne værdi skal ganges med to, da der går to stålstænger op ved taget:

\begin{center}
$7,\! 58kN \cdot 2 = 15,\! 1506888 kN$
\end{center}

Stålsystem eksklusiv tag:
\newline
Stålstyrken antages igen at være S235, som har densiteten $g=115\frac{kg}{m}$ \citep{Stabi}. For at finde kraften af en enkelt lodret stang multipliceres denne med højden af bygningen, som er 18 m, inklusiv kælderen.
\begin{center}
	$115\frac{kg}{m}\cdot 18m = 2070 kg$
\end{center}

\begin{center}
	$2070 kg \cdot 9,\!82\frac{m}{s^2} = 20,\! 33 kN$
\end{center}

Dette skal omregnes til at blive en linjelast og der divideres derfor med højden igen:

\begin{center}
$\frac{20,\! 33 kN}{18m} = 1,\! 13\frac{kN}{m}$
\end{center}

Det samme skal gøres for at finde kraften for en vandret stang, og densiteten skal her ganges med længden af den vandrette bjælke, som er 6 m.
\begin{center}
	$115\frac{kg}{m}\cdot 6m = 690 kg$
\end{center}

\begin{center}
	$690 kg \cdot 9,\!82\frac{m}{s^2} = 6,\!78  kN$
\end{center}

Dette skal omregnes til at blive en linjelast og der divideres derfor med længden af den vandrette stang:
$\frac{6,\! 78 kN}{6m} = 1,\! 13\frac{kN}{m}$
\newline
\newline
\textit{Opsamling}
\newline
Nu er alle egenlasterne fra taget, etagedækkene samt egenlasten fra stålsystemet fundet. Disse placeres nu i det statiskesytem vidst på Figur \ref{fig:dsadsd}

%Ikke det rigtige billede%
\begin{figure}[htbp]
	\centering
	\includegraphics[width=0.7\textwidth]{billeder/Tagmedvinkel.png}
	\caption{Dimensioner for tagetagen}
	\label{fig:tagetage}
\end{figure}

\subsection{Variable laster}
Af variable laster optræder der både snelast, vindlast og nyttelast på bygningen. Disse udregnes efter Dansk Standard Eurocode 1991.

\subsubsection{Snelast}
Til at beregne hvordan snelasten påvirker tilbygningen anvendes den karaktiske snelast og formlen:
\begin{center}
$s=\mu_iC_eC_ts_k$
\end{center}
\begin{itemize}
	\item[-] $s$: karakteristisk snelast
	\item[-] $\mu_i$: formfaktoren for snelasten, som sættes til 0.8 \citep[ tabel 5.2 kapitel 5.3]{EU91}
	\item[-] $C_e$: eksponeringsfaktoren
	\item[-] $C_t$: termisk faktor, som sættes til $1,\!0$ \citep[ kapitel 5.2]{EU91}
	\item[-] $s_k$: karakteristisk terrænværdi, som sættes til $1 \frac{kN}{m^2}$ \citep[ kapitel 4.1]{EU91}
\end{itemize}
Til at bestemme den karakteristiske snelast, beregnes eksponeringsfaktoren $C_e$.
\newline
\newline
Eksponeringsfaktoren, $C_e$, bestemmes ved:
\begin{center}
$C_e=C_{top}C_s$
\end{center}
\begin{itemize}
	\item[-] $C_{top}$: topografi faktor, som sættes til $1,\!0$ \citep[ tabel 5.1 kapitel 5.2]{EU91}
	\item[-] $C_s$: størrelse faktor, som sættes til $1,\!0$ \citep[ kapitel 5.2]{EU91}
\end{itemize}
Eksponeringsfaktoren kan nu bestemmes til:
\begin{center}
$C_e=1,\!0\cdot 1,\!0=1,\!0$
\end{center}
Strøybergs Palæ har et saddeltag, og dermed skal der tages højde for fire lasttilfælde, som ses på Figur \ref{fig:sne}. Derudover ganges snetilfældets værdi med 6,25 m, som er halvdelen af tilbygningens længde, hvilket gøres, da det statiske system er opdelt i tre rammer, for at få værdien ud i $\frac{kN}{m}$.

\begin{figure}[htbp]
	\centering
	\includegraphics[width=0.6\textwidth]{billeder/snelasttilfaelde.png}
	\caption{Fordeling af sne i de fire tilfælde \citep[ kapitel 5.3.3]{EU91}}
	\label{fig:sne}
\end{figure}

\underline{Snetilfælde 1}
\begin{center}
$s_1=0,\!8\cdot 1.0\cdot 1,\!0\cdot 1 \frac{kN}{m^2}\cdot 6,\!25 m=5,\!0 \frac{kN}{m}$
\end{center}
\underline{Snetilfælde 2 og 3}
\begin{center}
$s_2=\frac{1}{2}\cdot 0,\!8\cdot 1,\!0\cdot 1,\!0\cdot 1 \frac{kN}{m^2}\cdot 6,\!25 m=2,\!5 \frac{kN}{m}$
\end{center}
\underline{Snetilfælde 4}
\begin{center}
	$s_4=\mu_wC_eC_ts_k \frac{kN}{m^2}$
\end{center}
\begin{itemize}
	\item[-] $\mu_w$: formfaktoren, som sættes til $1,\!2$ eftersom $\alpha$ er $26,\!565^{\circ}$ \citep[ kapitel 5.3.3]{EU91}
\end{itemize}
Den karakteristiske snelast for lasttilfælde 4 kan nu bestemmes til:
\begin{center}
	$s_4=1,\!2\cdot 1,\!0\cdot 1,\!0\cdot 1 \frac{kN}{m^2}6,\!25 m=7,\!5 \frac{kN}{m}$
\end{center}
Til videre beregning har projektgruppen valgt at anvende snetilfælde 1, selvom der i praksis burde laves beregninger med samtlige snetilfælde.

\subsubsection{Vindlast}
For vindlasten regnes der en nettovindlast, hvilken regnes som summen af den udvendige og den indvendige vindlast.
\newline \indent{     }  For Strøybergs Palæ regnes der en nettovindlast for taget og for facaderne på bygningen, hvilket gøres for tre vindretninger, nord, øst og vest, da sydsiden af tilbygningen kommer i forlængelse af den allerede eksisterende bygning, og derfor formodes denne vindlast ikke at have en særlig betydning for tilbygningen.
\newline
\newline
Til at bestemme vindlasten på tilbygningens udvendige sider anvendes følgende formel:	
\begin{center} 
	$w_e=q_p(z_e)c_{pe}$
\end{center}
\begin{itemize}
	\item[-] $q_p$: peakhastighedstrykket
	\item[-] $z_e$: referencehøjden for det udvendige vindtryk, som sættes til 19 m for taget og 16 m for facaderne \ref{fig:tagetage}
	\item[-] $c_{pe}$: formfaktoren for det udvendige vindtryk
\end{itemize}

Til at bestemme vindlasten på tilbygningens indvendige sider anvendes følgende formel:
\begin{center} 
	$w_i=q_p(z_i)c_{pi}$
\end{center}
\begin{itemize}
	\item[-] $q_p$: peakhastighedstrykket
	\item[-] $z_i$: referencehøjden for det indvendige vindtryk, som sættes lig det udvendige for hhv. tag og facader $z_e$ \citep[ kapitel 7.2.9]{EU91}
	\item[-] $c_{pi}$: formfaktoren for det indvendige vindtryk
\end{itemize}

Nedenfor gives et beregningseksempel for beregning af $q_p$ for taget med vindretning fra vest. Beregninger for den udvendige peakhastighedstryk for facaderne samt for de to andre vindretninger, nord og øst, findes i bilag, hvor fremgangsmåden er identisk med nedenstående eksempel, blot med andre værdier fra de samme tabeller som anvendes i eksemplet, da vindretning og højden ændres. INDSÆTTE REFERENCE HER!
\newline
\newline
Den maksimale belastning fra vinden, peakhastighedstrykket $q_p$, bestemmes ved:
\begin{center}
$q_p(z_e)=[1+7I_v(z_e)]\frac{1}{2}pv_m^2(z_e)$
\end{center}
\begin{itemize}
	\item[-] $I_v$: vindturbulens
	\item[-] $\rho$: densiteten for luft $1,\!25 \frac{kg}{m^3}$ KILDE
	\item[-] $v_m$: middelvindhastigheden
\end{itemize}
For at bestemme peakhastigheden, beregnes først vindturbulens $I_v(z)$ samt middelvindhastigheden $v_m$.
\newline
\newline
Vindturbulens, $I_v(z)$, bestemmes ved:
\begin{center}
$I_v(z)=\frac{\sigma_v}{V_m(z)}=\frac{k_1}{c_0(z)\cdot ln(\frac{z}{z_0})}$
\end{center}
\begin{itemize}
	\item[-] $k_1$: turbulensfaktor, sættes til $1,\!0$ \citep[ kapitel 4.4]{EU91}
	\item[-] $c_0(z)$: orografifaktoren, som sættes til $1,\!0$ \citep[ kapitel 4.3.1]{EU91}
	\item[-] $z$: højden, som for taget er 19 m
	\item[-] $z_0$: ruhedslængde, som sættes til $1,\!0$ for terrænkategori IV \citep[ tabel 4.1 kapitel 4.3.2]{EU91}
\end{itemize}
Vindturbulensen kan nu bestemmes til:
\begin{center}
$I_v(z)=\frac{1,\!0}{1,\!0\cdot ln(\frac{19}{1,\!0})}=0,\!340$
\end{center}
Middelvindhastigheden, $v_m$, bestemmes ved:
\begin{center}
$v_m(z)=c_r(z)c_0(z)v_b$
\end{center}
\begin{itemize}
	\item[-] $c_r(z)$: ruhedsfaktor
	\item[-] $v_b$: basisvindhastigheden
\end{itemize}
Til at bestemme middelvindhastigheden, beregnes basisvindhastigheden samt ruhedsfaktor.
\newline
\newline
Basisvindhastigheden, $v_b$, bestemmes ved:
\begin{center}
$v_b=c_{dir}c_{season}v_{b,0}$
\end{center}
\begin{itemize}
	\item[-] $c_{dir}$: retningsfaktor, som sættes til $1,\!0$ ved vind fra vest \citep[ tabel 1a kapitel 4.2]{EU91}
	\item[-] $c_{season}$: årstidsfaktor, som sættes til $1,\!0$ \citep[ tabel 1b kapitel 4.2]{EU91}
	\item[-] $v_{b,0}$: grundværdi for basisvindhastigheden, som sættes til 24 $\frac{m}{s}$ \citep[ kapitel 4.2]{EU91}
\end{itemize}
Basisvindhastigheden kan nu bestemmes til:
\begin{center}
$v_b=1,\!0\cdot 1,\!0\cdot 24 \frac{m}{s}=24 \frac{m}{s}$
\end{center}
Ruhedsfaktor, $c_r(z)$, bestemmes ved:
\begin{center}
$c_r(z)=k_rln(\frac{z}{z_0})$
\end{center}
\begin{itemize}
	\item[-] $k_r$: terrænfaktor
\end{itemize}
Terrænfaktoren, $k_r$, bestemmes ved:
\begin{center}
$k_r=0,\!19\cdot (\frac{z_0}{z_{0,II}})^{0,\!07}$
\end{center}
\begin{itemize}
	\item[-] $z_{0,II}$: værdi for ruhedslængde for terrænkategori II, som sættes til $0,\!05$ \citep[ kapitel 4.3.2]{EU91}
\end{itemize}
\begin{center}
$k_r=0.19\cdot (\frac{1,\!0}{z_{0,\!05}})^{0,\!07}=0,\!234$
\end{center}
Ruhedsfaktor kan nu bestemmes til:
\begin{center}
$c_r(z)=0,\!234\cdot ln(\frac{19}{1.0})=0,\!690$
\end{center}
Middelvindhastigheden kan nu bestemmes til:
\begin{center}
$v_m(z)=0,\!690\cdot 1,\!0\cdot 24 \frac{m}{s}=16,\!569 \frac{m}{s}$
\end{center}
Peakhastighedstrykket $q_p$ i højden z, kan nu bestemmes til:
\begin{center}
$q_p(z_e)=[1+7\cdot 0,\!340]\cdot \frac{1}{2}\cdot 1,\!25 \frac{kg}{m^3}\cdot (16,\!569 \frac{m}{s})^2=0,\!579 \frac{kN}{m^2}$
\end{center}

På tabellen nedenfor ses værdierne for peakhastighedstrykket for alle tre vindretninger, og ved højderne 16 m og 18 m.
\begin{table}[htb]
\begin{center}
	\begin{tabular}{ |c|c|c| } 
		\hline
		Vindretning/Højden & $q_p [\frac{kN}{m^2}]$ ved 16 m & $q_p [\frac{kN}{m^2}]$ ved 19 m \\	\hline
		Vest & 0,536 & 0,579 \\		\hline
		Øst & 0,428 & 0,463 \\	\hline 
		Nord & 0,428 & 0,463 \\ 	\hline
	\end{tabular}
		\caption{Værdier for $q_p [\frac{kN}{m^2}]$}
		\label{tab:peak}
\end{center}
\end{table}

Peakhastighedstrykket skal nu anvendes, for at bestemme vindlasten på taget fra de tre vindretninger.
\newline
\newline
\underline{Vindtryk på tag}
\newline
Først bestemmes de karakteristiske vindlaster, som skal anvendes i lastkombinationerne, ved opstilles en række vindtilfælde for henholdsvis tryk og sug. 
\newline
\newline
For vindlasten på taget skal der regnes en vindkraft ud for den udvendige vindlast og den indvendige vindlast. Det antages, at Strøybergs Palæ har sadeltag, og det anbefales derfor at dele taget ind i zoner efter det nationale anneks. Bygningen tegnes i fuld størrelse, og derfor er den nuværende bygning også medregnet i målene. For vindretningen $\theta = 0^{\circ}$ (gælder for vindretning fra vest og fra øst) skal bygningen deles ind som vist på Figur \ref{fig:opdeling}.  

\begin{figure}[htbp]
	\centering
	\includegraphics[width=0.8\textwidth]{billeder/opdeling.png}
	\caption{Opdeling af bygning, for vindretning fra øst og vest \citep[ kapitel 7.2.5]{EU91}}
	\label{fig:opdeling}
\end{figure}

Formfaktorerne for tagets zoner, $c_{pe}$ bestemmes ud fra Eurocode 1991. Her anvendes taghældningen på $26,\!56^{\circ}$, og der laves derfor lineær regression mellem $c_{pe}$ værdierne for vinklerne 15 og 30 grader, efter anbefaling af det nationale anneks \citep[ tabel 7.4a kapitel 7.2.5]{EU91}. For $\theta = 0^{\circ}$ skifter trykket hurtigt mellem positive og negative værdier i vindsiden, ved en taghældning mellem $\alpha = -5^{\circ} til + 45^{\circ}$, og derfor skal der regnes for både positive og negative formfaktorværdier. 
\newline \indent{     }  Nedenfor er et beregningseksempel for beregning af formfaktoren.
\newline
\newline
\underline{Zone F for vind fra vest}
\newline
Ud fra \citep[ tabel 7.4a kapitel 7.2.5]{EU91} er de negative værdier for zone F: $-0,\!9$ og $-0,\!5$. Her ud fra fås ligningen, og $c_{pe,10,neg}$ bestemmes:
\begin{center}
	$f(\alpha)=0,\!0267\alpha - 1,\!3 \to c_{pe,10,neg}=-0,\!592$
\end{center}
De positive værdier for zone F er: $0,\!2$ og $0,\!7$. Her ud fra fås ligningen, og $c_{pe,10,pos}$ bestemmes:
\begin{center}
	$f(\alpha)=0,\!0333\alpha - 0,\!3 \to c_{pe,10,pos}=0,\!585$
\end{center}

De beregnede værdier for formfaktorerne ses i Tabellerne; \ref{tab:cc} og \ref{tab:kk}. 
\newline
\newline
De udvendige vindtryk beregnes ved formlen, som tidligere vist:
\begin{center} 
	$w_e=q_p(z_e)c_{pe}$
\end{center}
Disse værdier findes alle i Bilag - HUSK REFERENCE! 
\newline
\newline
De indvendige vindtryk skal nu bestemmes, da de virker på samme tid som de udvendige vindtryk. Den indre vindlast virker dog altid i samme retning på alle indre overflader for en konstruktion. Der kan derfor kun optræde enten tryk eller sug, og aldrig begge tilfælde på forskellige indvendige flader.  
\newline
\newline
Der findes to måder til bestemmelse af formfaktoren, $c_{pi}$, for indvendig vind. I dette projekt anvendes den forsimplede metode, hvor $c_{pi}$ regnes til at være den mindst gunstige af + 0,2 og - 0,3 \citep[Kapitel 7]{EU91}. Værdierne for formfaktoren ses i Tabellerne \ref{tab:cc} og \ref{tab:kk}.
\newline
\newline
Herefter kan de indvendige vindtryk beregnes ved formlen, som tidligere vist:
\begin{center} 
	$w_e=q_p(z_e)c_{pi}$
\end{center}

\begin{table}[htb]
	\begin{center}
		\begin{tabular}{ |c|c|c|c|c| } 
			\hline
			Zone/Formfaktor & \multicolumn{2}{l|}{$c_{pe,10 udvendig}$} & \multicolumn{2}{l|}{$c_{pe,10 indvendig}$} \\	\hline
			& Positiv & Negativ & Positiv & Negativ \\ \hline
			F & 0,585 & -0,592 & 0,200 & -0,300 \\	\hline
			G & 0,585 & -0,569 & 0,200 & -0,300 \\	\hline 
			H & 0,354 & -0,223 & 0,200 & -0,300 \\ 	\hline
			I & 0 & -0,400 & 0,200 & -0,300 \\	\hline
			J & 0 & -0,615 & 0,200 & -0,300 \\	\hline
		\end{tabular}
		\caption{Værdier for $c_{pe,10}$ på udvendige og indvendige tagoverflader for vind fra vest og øst, vindretning $0^{\circ} = 180^{\circ}$ på taget}
		\label{tab:cc}
	\end{center}
\end{table}

\begin{table}[htb]
	\begin{center}
		\begin{tabular}{ |c|c|c|c| } 
			\hline
			Zone/Formfaktor & $c_{pe,10 udvendig}$ & \multicolumn{2}{l|}{$c_{pe,10 indvendig}$} \\	\hline
			& Negativ & Positiv & Negativ \\ \hline
			F & -1,146 & 0,200 & -0,300 \\	\hline
			G & -1,377 & 0,200 & -0,300 \\	\hline 
			H & -0,754 & 0,200 & -0,300 \\ 	\hline
			I & -0,500 & 0,200 & -0,300 \\	\hline
		\end{tabular}
		\caption{Værdier for $c_{pe,10}$ på udvendige og indvendige tagoverflader for vind fra nord, vindretning $90^{\circ}$ på taget}
		\label{tab:kk}
	\end{center}
\end{table}

For at bestemme nettovindtrykket skal der opstilles en række tilfælde for, hvordan det udvendige vindtryk og det indvendige vindtryk kan optræde.  
\newline
\newline
Der opstilles fire vindkombinationer for hver vindretning:
\begin{enumerate}
	\item Tryk udvendigt på FGH + sug for JI og invendigt sug
	\item Tryk udvendigt på FGH + sug for JI og invendigt tryk
	\item Sug udvendigt for FGH + tryk på JI og invendigt sug
	\item Sug udvendigt for FGH + tryk på JI og invendigt tryk
\end{enumerate}

Nedenfor laves et beregningseksempel med vind fra vest med vindkombination 1 for zone F. De resterende værdier kan ses i Tabel \ref{tab:bb}.
\newline
\newline
Enheden skal være $[\frac{kN}{m}]$, og derfor multipliceres med længden mellem hver af vores stænger, som er 6,25 m. Se Figur \ref{fig:farvel}.

\begin{center} 
	$w_F=(w_{e,F}-w_{i,F})\cdot 6,\!25 m$
\end{center}

\begin{itemize}
	\item[-] $w_{e,F}$: udvendig vindtryk for zone F = $0,\!339 \frac{kN}{m^2}$
	\item[-] $w_{i,F}$: indvendig vindtryk for zone F = $-0,\!174 \frac{kN}{m^2}$
\end{itemize}

\begin{center} 
	$w_F=(w_{e,F}-w_{i,F})\cdot 6,\!25 m = (0,\!339 \frac{kN}{m^2}\cdot (-0,\!174 \frac{kN}{m^2})\cdot 6,\!25 m = 3,\!203 \frac{kN}{m}$
\end{center}

For alle vindretningerne anvendes vindkombination 1 til videre beregning, og disse laster ses i tabellen nedenfor! 

\begin{table}[htb]
	\begin{center}
		\begin{tabular}{ |c|c|c|c| } 
			\hline
			Zone/Vindretning, w $[\frac{kN}{m}]$ & Vest & Øst & Nord \\	\hline
			F & 3,203 & 2,563 & -2,445 \\	\hline
			G & 3,203 & 2,563 & -3,804 \\	\hline 
			H & 2,367 & 1,893 & -1,314 \\ 	\hline
			I & -0,362 & -0,289 & -0,579 \\	\hline
			J & -1,138 & -0,910 & - \\	\hline
		\end{tabular}
		\caption{Værdier for nettovindtryk på taget}
		\label{tab:bb}
	\end{center}
\end{table}

Fra Tabel \ref{tab:bb} ses, at værdierne for vest er større i forhold til værdierne for øst. Det formodes derfor, at vindlasten er mere kritisk ved vind fra vest, og der ses fremover bort fra vinden fra øst i projektet. I tabellen ses det ligeledes, at værdierne for nord alle er negative, hvilket skyldes at vinden fra nord kun kan påvirke taget med sug, og dermed ses der i denne rapport også bort fra denne. 
\newline \indent{     }  Derfor anvendes værdierne for vest til de videre beregninger, når der skal opstilles lastkombinationer vindlasten for taget.
\newline
\newline
\underline{Vindtryk på facaderne}
\newline
Vinden vil også påvirke Strøybergs Palæ på dens facader og endegavle, og derfor skal vindlasten også beregnes for de lodrette vægge af bygningen.
\newline \indent{     }  Nettovindtrykket på facaderne beregnes på samme måde som nettovindtrykket for taget, dog tages der udgangspunkt i \citep[ tabel 7.1]{EU91}. Længden af bygningens facader skal anvendes, og igen bruges den fulde længde af bygningen, der består af den nuværende bygning samt tilbygningen, hvilket giver længden 25 m. 
\newline
\newline
For vindretning $\Theta = 0^{\circ}$ for vind fra vest og øst gælder at b = 25 m, d = 12 m og højden h = 16 m på Figur \ref{fig:vindvest} og \ref{fig:vindost}.

\begin{figure}[htbp]
	\centering
	\includegraphics[width=0.5\textwidth]{billeder/vindvest1.png}
	\caption{Vind på facaden fra vest \citep[ 7.2.2]{EU91}}
	\label{fig:vindvest}
\end{figure}

\begin{figure}[htbp]
	\centering
	\includegraphics[width=0.7\textwidth]{billeder/vestost.png}
	\caption{Zoner for vind fra vest og øst på facaderne \citep[ 7.2.2]{EU91}}
	\label{fig:ab}
\end{figure}

Bygningen skal igen opdeles i nogle zoner, både for vind fra vest og øst, som kan ses på Figur \ref{fig:ab}, og for vind fra nord, hvilket ses på Figur \ref{fig:ac}.

\begin{figure}[htbp]
	\centering
	\includegraphics[width=0.5\textwidth]{billeder/vindost1.png}
	\caption{Vind på facaden fra øst \citep[ 7.2.2]{EU91}}
	\label{fig:vindost}
\end{figure}

\begin{figure}[htbp]
	\centering
	\includegraphics[width=0.7\textwidth]{billeder/nord.png}
	\caption{Zoner for vind fra nord på gavlen \citep[ 7.2.2]{EU91}}
	\label{fig:ac}
\end{figure}

Igen regnes der først en udvendige vindlast på facaderne, og derefter en indvendig vindlast på facaderne, inden der opstilles en række vindkombinationer, hvor nettovindtrykket udregnet. For vindkombinationerne ses der dog bort fra siderne A, B og C, da vores ramme ligger 6,25 meter inde i konstruktionen, og derfor ikke har zonerne A, B og C:
\begin{enumerate}
	\item Tryk udvendigt på D og sug indvendigt
	\item Sug udvendigt på E og sug indvendigt
	\item Tryk udvendigt på D og tryk indvendigt  
	\item Sug udvendigt på E og tryk indvendigt
\end{enumerate}

Der arbejdes videre med vindkombination 1, hvor den udvendige vind virker som tryk på siden D og sug på siden E og med en indre vindlast som sug.
\newline \indent{     }  Værdierne for nettovindtrykket kan ses på Figur \ref{tab:hh}.

\begin{table}[htb]
	\begin{center}
		\begin{tabular}{ |c|c|c|c| } 
			\hline
			\multirow{2}{*}{Zone/Formfaktor} & \multirow{2}{*}{$c_{pe,10}$ udvendig} & \multicolumn{2}{l|}{$c_{pe,10}$ indvendig} \\ \cline{3-4} 
			& & Positiv & Negativ   		\\ \hline
			A & -1,200 & 0,200 & -0,300 \\	\hline
			B & -0,800 & 0,200 & -0,300 \\	\hline 
			D & 0,800 & 0,200 & -0,300 \\	\hline
			E & -0,517 & 0,200 & -0,300 \\	\hline
		\end{tabular}
		\caption{Værdier for $c_{pe,10}$ for vind fra vest og øst}
		\label{tab:ff}
	\end{center}
\end{table}

\begin{table}[htb]
	\begin{center}
		\begin{tabular}{ |c|c|c|c| } 
			\hline
			\multirow{2}{*}{Zone/Formfaktor} & \multirow{2}{*}{$c_{pe,10}$ udvendig} & \multicolumn{2}{l|}{$c_{pe,10}$ indvendig} \\ \cline{3-4} 
			& & Positiv & Negativ   		\\ \hline
			A & -1,200 & 0,200 & -0,300 \\	\hline
			B & -0,800 & 0,200 & -0,300 \\	\hline
			C & -0,500 & 0,200 & -0,300 \\	\hline 
			D & 0,752 & 0,200 & -0,300 \\	\hline
			E & -0,404 & 0,200 & -0,300 \\	\hline
		\end{tabular}
		\caption{Værdier for $c_{pe,10}$ for vind fra nord}
		\label{tab:gg}
	\end{center}
\end{table}

\begin{table}[htb]
	\begin{center}
		\begin{tabular}{|c|c|c|c|}
			\hline
			Zone/Vindretning, w $[\frac{kN}{m}]$ & Vest & Øst & Nord \\ \hline
			D & 3,686 & 2,946 & 2,142 \\ \hline
			E & -0,726 & -0,580 & -0,953 \\ \hline
		\end{tabular}
		\caption{Værdier for nettovindtryk på siderne}
		\label{tab:hh}
	\end{center}
\end{table}

Ud fra Tabel 6.7 ses det, at værdierne for vest er størst. Det formodes igen, at vindlasten derfor er mere kritisk fra vest end for de to andre vindretninger, og der ses fremover bort fra vinden fra nord og øst. Derfor anvendes lasterne for vest videre i lastkombinationerne, når der skal opstilles lastkombinationer for siden af bygningen.
\newline \indent{     }  Disse linjelaster virker på højden af facaderne, der er siderne D og E, som hver er 16 m høje.
\newline
\newline
\underline{Opsummering}
\newline
Det er nu bestemt i denne rapport, at der fokuseres på vind fra vest, hvor der optræder forskellige udvendige laster, mens den indre altid optræder som sug. Lasterne er omregnet til linjelaster ved at gange med afstanden mellem vores rammer, hvilken er 6,25 m. 
\newline
\newline
Systemet betragtes, hvor der laves beregninger for rammen i midten af konstruktionen, hvilken ligger 6,25 m inde på konstruktionen. Det betyder, rammen ligger mellem zone F og G (se Figur \ref{fig:opdeling}). Derfor udregnes et gennemsnit mellem værdierne for zonerne F og G ved vind på taget, da zonen E for siderne gælder for hele facaden. 
Dog ses det, at værdierne F og G for taget er ens, og dermed er gennemsnitsværdien lig med værdierne for F og G. 

\textbf{Nyttelast}
Ud fra \citep[ tabel 6.2 kapitel 6.3.1.2]{EU91} aflæses den jævnt fordelte last, $q_k$, for kategori A1, som er bolig og lokale adgangsveje, til at være $1,\!5 \frac{kN}{m^2}$. Denne last beregnes for alle etager på tilbygningen samlet set. Denne omregnes til en linjelast ved at multiplicere med afstanden mellem rammerne på $6,\!25$ m. Hermed fås en linjelast på $9,\!37 \frac{kN}{m}$. 

\section{Lastkombinationer}
Den karakteristiske egenlast, jordlast, vindlast, snelast og nyttelast er bestemt for tilbygningen til Strøybergs Palæ. Derfor opstilles derfor en række forskellige lasttilfælde, for at beregne de regningsmæssige laster, som skal anvendes til beregningerne af reaktioskræfter for det statiske system af tilbygningen til Strøybergs Palæ.
\newline \indent{     }  Lastkombinationerne opstilles ved at betragte tilbygningen i tre dele; Taget, rammen mellem jorden og taget samt rammen under jorden. Disse deles ind i områder, da lasterne ikke virker ens på alle konstruktionsdele, hvilket er illustreret på Figur (INDSÆT FIGUR AF OMRÅDER PÅ STATISK SYSTEM).
\newline
\newline
Ved et rigtigt byggeprojekt skal der opstilles lasttilfælde for alle tænkelige scenarier, som kan forekomme. For dette projekt er det dog ikke en mulighed, og derfor fokuseres der kun udvalgte tilfælde.
\newline
\newline
Formlen for den regningsmæssige last er:
\begin{center}
	$E_d = \gamma_{G1} K_{FI} G_{K1} + \gamma_{Q1} K_{FI} Q_{K1} + \gamma_{Q2} \Psi_{0,2} K_{FI} Q_{K2} + \gamma_{Q3} \Psi_{0,3} K_{FI} Q_{K3}$ 
\end{center}

\begin{itemize}
	\item[-] $\gamma_G$: Partialkoefficient for permanent last
	\item[-] $K_{FI}$: Konsekvensklasse
	\item[-] $G_K$: Karakteristisk værdi for permanent last
	\item[-] $\gamma_Q$: Partialkoefficient for variabel last
	\item[-] $Q_K$: Karakteristisk værdi for variabel last
	\item[-] $\Psi$: Kombinationsværdi af variabel last. Mulitipliceres den værdi, som ikke er dominerende 
\end{itemize}

Der er lavet tre lasttilfælde for områder 1; for henholdsvis egenlasten, snelasten og vindlasten dominerende. Snelasten dominerende er beregnet ved: VENT MED AT INDSÆTTE TIL DET ER GODKENDT AF JOHAN. 
\newline
\newline
Nedenstående tabel viser resultaterne for hvert område, med skiftende dominerende laster.
\newline
\newline
INDSÆT TABEL MED VÆRDI FOR LASTTILFÆLDE!
\newline
\newline