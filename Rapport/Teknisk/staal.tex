\chapter{Brudgrænsetilstand}

I dette afsnit undersøges tilbygningens brudgrænsetilstand. Først beregnes reaktionerne ud fra Figur \ref{fig:alle} og dernæst laves snitkræfter. Slutteligt undersøges spændingstilstanden, og udfra ståltypens flydespændingen kan det vurderes, om konstruktionen kan holde eller vil bryde sammen.   

\section{Reaktioner}
Når der beregnes reaktioner for de tre understøtninger, er der behov for at opdele konstruktionen. Systemet opdeles i det midterste charnierled i henholdvis en venstre- og højre del, som ses på Figur \ref{fig:opdelingv} og \ref{fig:opdelingh}. I skæringen mellem de to dele, vil der være snitkræfter, men ikke momentkræfter, da momentet i charnierledet er nul. Dermed optræder der kun normalkraften, N, og forskydningskraften, V.

\begin{figure}[H]
	\centering
	\includegraphics[width=0.7\textwidth]{billeder/venstre.png}
	\caption{Venstre side af systemet med reaktioner, samt fritlegemediagram}
	\label{fig:opdelingv}
\end{figure}

\begin{figure}[H]
	\centering
	\includegraphics[width=0.7\textwidth]{billeder/hojre.png}
	\caption{Højre side af systemet, samt fritlegemediagram}
	\label{fig:opdelingh}
\end{figure}

Reaktionerne på højre side af systemet, der virker i charnieret, sættes som belastninger på venstre del af systemet. Reaktionerne på højre del af systemet beregnes først.
\newline
\newline
Først bestemmes den vandrette reaktion i charnier ledet, $S_v$ gennem vandret ligevægt: 
\begin{center}
	$\rightarrow+:0 = F_x + E_x - J(2m) - V_k \cdot 2m - V_2 \cdot 16m$
	\newline
	$E_x = -1309,\!49 N$
\end{center}

Nu bestemmes $C_y$ gennem moment ligevægt om punkt E: 
\begin{center}
	$\AR{}:0 = -F_y \cdot 6m - E_1 \cdot 18m \cdot 6m + C_y \cdot 6m - J(2m) \cdot 17,\!33m - V_k \cdot 2m \cdot 17m - E_s \cdot 6m \cdot 3m - V_2 \cdot 16m \cdot 8m$
	\newline
	$C_y = 7,\!80 \cdot 10^5 N$
\end{center}

Nu kan den lodrette reaktion i charnier ledet, $S_l$ bestemmes gennem lodret ligevægt: 
\begin{center}
	$\uparrow+: 0 = F_y - E_1 \cdot 18m - E_s \cdot 6m + C_y + S_l$
	\newline	
	$E_y = -2,\!44 \cdot 10^5N$
\end{center}

Reaktionerne for det højre system er hermed bestemt, og disse påsættes som belastninger i punkt E i det venstre system, og reaktionerne bestemmes for dette. Da reaktionerne i begge systemer peger i negativ retning, ændres fortegnet og dermed retning, så de virker som vist på Figur \ref{fig:opdelingv}.

Først tages der moment om A, for at beregne $B_y$  
\begin{center}
	$A\hookrightarrow+: 0 = B_y \cdot 6m + E_y \cdot 6m + E_x \cdot 18m - E_2 \cdot 18m \cdot 6m - E_s \cdot 6m \cdot 3m + D_x \cdot 18m - V_1 \cdot 16m \cdot (8m + 2m) - J(2m) \cdot (2m \cdot \frac{1}{3}) - V_k \cdot 2m \cdot 1m$
	\newline 
	$B_y = 9,\!49 \cdot 10^5N$
\end{center}

Nu laves lodret ligevægt for at bestemme $A_y$
\begin{center}
	$\uparrow+: 0 = A_y + B_y - E_y - E_2 \cdot 18m - E_1 \cdot 18m - F_y - E_s \cdot 6 m$
	\newline
	$A_y = 5,\!35 \cdot 10^5N$
\end{center}

For at gøre det muligt at isolere en af de vandrette reaktioner laves der et snint i charnieret, og derefter tages der moment omkring charnieret i punkt E, for at bestemme $B_x$.
\begin{center}
	$\hookrightarrow+: 0 = B_x \cdot 18m$
	\newline
	$B_x = 0N$
\end{center}

Slutteligt laves vandret ligevægt for at bestemme $A_x$
\begin{center}
	$\rightarrow+: 0 = A_x - E_x + B_x + J(2m) + V_k \cdot 2m + V_1 \cdot 16 m - D_x$
	\newline
	$A_x = -1,\!33 \cdot 10^5N$
\end{center} 

Alle reaktionerne er hermed bestemt, og der kan laves snitkræfter.

\section{Snitkræfter}
Snitkræfterne for Strøybergs Palæs stålramme kan nu beregnes gennem de kendte reaktioner. Her laves der snitkræfter for to tilfælde; brudgrænse- og anvendelsesgrænsetilstand. I beregningen af brudgrænsetilstanden skal der ses på de mest ekstreme lasttilfælde (kritiske moment, normal og forskydning?) for at se, om stålkonstruktionen kan holde til disse belastninger. For anvendelsesgrænsetilstanden skal der undersøges, hvor store deformationer stålrammerne får ved lasterne, og om disse overskrider de maksimalt tilladte udbøjninger jævnført Eurocode 1993, afsnit 7.2. Her kan der dog gennem Eurocode 1993, afsnit 7.2 tages nogle forbehold, som gør at konstruktionen ikke er nær så belastet, som ved brudgrænsetilstanden, da der kun skal fokuseres på én variabel last og ikke alle laster, som optræder i konstruktionen; der beregnes derfor kun ud fra vindlasten. Det betyder også, at der skal udregnes nye reaktioner for systemet, hvor vindlasten optræder som eneste variable last.
\newline
\newline
For begge tilstande fokuseres der kun på hovedkonstruktionen og ikke taget, og der tages derfor udgangspunkt i rammerne for henholdsvis brudgrænse- og anvendelsesgrænsetilstand, som illustreret på figur NYE BILLEDER – FIGUR 1 og 2, da disse er grundkonstruktionen, hvor belastning fra taget optræder i punkterne D og F.

\begin{figure}[H]
	\centering
	\includegraphics[width=0.8\textwidth]{billeder/snitbrud.png}
	\caption{Snit}
	\label{fig:snitbrud}
\end{figure}

\section{Spænding}
For at finde ud af om konstruktionen kan holde, undersøges spændingstilstanden. Her skal det gælde:

\begin{center}
	$\sqrt{\sigma^2 + 3\tau^2} \leq f_y$ 
\end{center}

\begin{itemize}
	\item[-] $\sigma$: Normalspænding
	\item[-] $\tau$: Forskydningsspænding
	\item[-] $f_y$: Flydespænding
\end{itemize}

Flydespændingen beregnes ved formlen:

\begin{center}
	$f_y = \frac{f_{yk}}{\gamma}$
\end{center}

\begin{itemize}
	\item[-] $f_{yk}$: Den karakteristiske flydespænding, der afhænger af ståltype. For ståltype S235 er $f_{yk} = 225 MPa$
	\item[-] $\gamma$: Partialkoefficient, der sættes til 1,1 \citep[ s. 212]{stabi}.  
\end{itemize}

Altså beregnes den regningsmæssige flydespænding til:

\begin{center}
	$f_y = \frac{225 MPa}{1,\!1} = 204,\!54 MPa$
\end{center}

Normalspændingen findes ved Naviers formel:

\begin{center}
	$\sigma = \frac{N}{A} - \frac{M}{I} y$
\end{center}

\begin{itemize}
	\item[-] N: Normalkraft [kN]
	\item[-] A: Tværsnitsareal, som for stålprofil 450 er $14,\!7 \cdot 10^3 mm^2$ \citep{stabi}. 
	\item[-] M: Moment [kN]
	\item[-] y: Tyngdepunktskoordinat [mm]
	\item[-] I: Inertimoment, som for stålprofil 450 er $458,\!5 \cdot 10^6 mm^4$ \citep{stabi}. 
\end{itemize} 

Dernæst beregnes forskydningsspændingen ved Grasshofs formel:

\begin{center}
	$\tau = \frac{VQ}{Ib}$
\end{center}

\begin{itemize}
	\item[-] V: Forskydningskraft [kN]
	\item[-] Q: 1. ordens arealmoment for $A_1$: $Q = \int_{A_1}y \mathrm{d}A = yA$ $[mm^3]$
	\item[-] b: bredde
\end{itemize}

Nedenfor vises et beregningseksempel for et kritisk punkt. Resultaterne for alle de valgte kritiske punkter kan ses i tabel XX. 
\newline
\newline
VIS BEREGNINGSEKSEMPEL FOR ET KRITISK PUNKT
\newline
\newline
INDSÆT TABEL MED RESULTATER (SE MAPLE-DOKUMENT) 
\newline
\newline
SKRIV KONKLUSION (KAN DET HOLDE)? Hvad kan man gøre hvis det ikke kan holde - eller er det "overdimensioneret"?