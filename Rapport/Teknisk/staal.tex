\chapter{Brudgrænsetilstand}
I dette afsnit undersøges tilbygningens brudgrænsetilstand. Først beregnes reaktionerne ud fra Figur \ref{fig:alle} og dernæst laves snitkræfter. Slutteligt undersøges spændingstilstanden, og udfra ståltypens flydespændingen kan det vurderes, om konstruktionen kan holde eller vil bryde sammen.   

\section{Reaktioner}
Reaktionerne i de tre understøtninger beregnes ved at opdele konstruktionen. Systemet opdeles i det midterste charnierled i henholdvis en venstre- og højre del, som ses på Figur \ref{fig:opdelingv} og \ref{fig:opdelingh}. I skæringen mellem de to dele, vil der være snitkræfter; dog er bøjningsmomentet \textit{M} lig 0 i charnieret. Dermed optræder der kun normalkraften, \textit{N}, og forskydningskraften, \textit{V}. Det er derfor muligt, at betragte højre del som en isoleret del af konstruktionen med fast, simpel understøtning i punkt E.
\\
\\
Inden reaktionerne bestemmes bestemmes jordlastens resultant, j. Dette gøres ved et bestemt integrale: 
\begin{equation}
j = \int_{0}^{2} \! j(z) \, \mathrm{d}z \leftrightarrow j = 127,\!95 kN
\end{equation}

\begin{figure}[H]
	\centering
	\includegraphics[width=0.7\textwidth]{billeder/hojre.png}
	\caption{Højre side af systemet, samt fritlegemediagram}
	\label{fig:opdelingh}
\end{figure}

\begin{figure}[H]
	\centering
	\includegraphics[width=0.7\textwidth]{billeder/venstre.png}
	\caption{Venstre side af systemet, samt fritlegemediagram}
	\label{fig:opdelingv}
\end{figure}

Reaktionerne i punkt E kan sættes som punktlaster i charnieret på den venstre side af systemet, som ses på Figur \ref{fig:opdelingv}. Reaktionerne på højre del af systemet beregnes først.
\newline
\newline
Først bestemmes den vandrette reaktion i charnieret, $E_x$ gennem vandret ligevægt: 
\begin{equation}
\begin{split}
	+\atop\rightarrow: 0 = & F_x + E_x - j - v_k \cdot \SI{2}{m} - v_2 \cdot \SI{16}{m}
	\\ &
	E_x = \SI{62,52}{kN}
\end{split}
\end{equation}

$C_y$ bestemmes gennem momentligevægt om punkt E: 
\begin{equation}
\begin{split}
	\ALN{\text{E}} :0 = & -F_y \cdot \SI{6}{m} - E_1 \cdot \SI{18}{m} \cdot \SI{6}{m} + C_y \cdot \SI{6}{m} - j \cdot \SI{17,33}{m} - v_k \cdot \SI{2}{m} \cdot \SI{17}{m} \\ & - E_s \cdot \SI{6}{m} \cdot \SI{3}{m} - v_2 \cdot \SI{16}{m} \cdot \SI{8}{m}
	\\ &
	C_y = \SI{964,68}{kN}
\end{split}
\end{equation}

Til sidst bestemmes den lodrette reaktion i charnierledet, $E_y$ gennem lodret ligevægt: 
\begin{equation}
\begin{split}
	\uparrow+: 0 = & F_y - E_1 \cdot \SI{18}{m} - E_s \cdot \SI{6}{m} + C_y + E_y
	\\ &
	E_y = \SI{-349,74}{kN}
\end{split}
\end{equation}

Reaktionerne $E_y$ og $E_x$ påsættes som belastninger i punkt E på det venstre system, så de virker som vist på Figur \ref{fig:opdelingv}. Reaktionerne i det venstre system kan hermed bestemmes.
\\
\\
Først tages der moment om A, for at beregne $B_y$:
\begin{equation}
\begin{split}
	\ALN{\text{A}}: 0 = & B_y \cdot \SI{6}{m} + E_y \cdot \SI{6}{m} + E_x \cdot \SI{18}{m} - E_2 \cdot \SI{18}{m} \cdot \SI{6}{m} - E_s \cdot \SI{6}{m} \cdot \SI{3}{m} \\ & + D_x \cdot \SI{18}{m} - v_1 \cdot \SI{16}{m} \cdot (\SI{8}{m} + \SI{2}{m}) - j \cdot (\SI{2}{m} \cdot \frac{1}{3}) - v_k \cdot \SI{2}{m} \cdot \SI{1}{m}
	\\ &
	B_y = \SI{658,26}{kN}
\end{split}
\end{equation}

Nu laves lodret ligevægt for at bestemme $A_y$:
\begin{equation}
\begin{split}
	\uparrow+: 0 = & A_y + B_y - E_y - E_2 \cdot \SI{18}{m} - E_1 \cdot \SI{18}{m} - F_y - E_s \cdot \SI{6}{m}
	\\ &
	A_y = \SI{732,77}{kN}
\end{split}
\end{equation}

For at gøre det muligt at isolere en af de vandrette reaktioner laves der et snit i charnieret. Derefter betragtes den nedre del, og der tages moment omkring charnieret i punkt E, for at bestemme $B_x$:
\begin{equation}
\begin{split}
	\ALN{\text{E}}: 0 = & B_x \cdot \SI{18}{m}
	\\ &
	B_x = \SI{0}{kN}
\end{split}
\end{equation}

Slutteligt laves vandret ligevægt for at bestemme $A_x$:
\begin{equation}
\begin{split}
	+\atop\rightarrow: 0 = & A_x - E_x + B_x + j + v_k \cdot \SI{2}{m} + v_1 \cdot \SI{16}{m} - D_x
	\\ &
	A_x = \SI{-133,31}{kN}
\end{split}
\end{equation} 

\section{Snitkræfter}
Snitkræfterne for stålrammen til tilbygningen til Strøybergs Palæs kan nu beregnes. Her laves der syv snit, som er illustreret på Figur \ref{fig:snitbrud}. Disse resultater vil bruges til at undersøge, om konstruktionen har en tilstrækkelig bæreevne, eller om den vil bryde sammen. 

\begin{figure}[H]
	\centering
	\includegraphics[width=0.8\textwidth]{billeder/snitbrud.png}
	\caption{Snit}
	\label{fig:snitbrud}
\end{figure}

Nedenfor vises beregningseksempel for snit 1 og snit 5. Beregningerne for de resterende snit kan ses i Bilag ???. 
\newline
\newline
\textbf{Snit 1: 0 m < x < 2 m}
\newline
Fritlegemediagrammet for snit 1 ses på Figur \ref{fig:snitet}.
\newline
\newline
Først bestemmes normalkraften:
\begin{equation}
	0 = N_1 + A_y - E_1 \cdot x \leftrightarrow N_1(x) = 31,\!54 \frac{\text{kN}}{\text{m}} x - \SI{547,95}{kN}
\end{equation}

Nu bestemmes forskydningskraften:
\begin{equation}
	0 = V_1 + j + v_k \cdot x + A_x \leftrightarrow V_1(x) = -33,\!64\frac{\text{kN}}{\text{m}} x + \SI{133,31}{kN}
\end{equation}

Til sidst bestemmes moment:
\begin{equation}
	0 = M_1 + j \cdot \frac{2x}{3} + v_k \cdot x \cdot \frac{x}{2} + A_x \cdot x \leftrightarrow M_1(x) = -22,\!15\frac{\text{kN}}{\text{m}} x^2 + \SI{133,30}{kN} x
\end{equation}

\begin{figure}[H]\centering
	\begin{minipage}[b]{0.48\textwidth}\centering
		\includegraphics[width=0.80\textwidth]{billeder/snitet.png} %Venstre billede
	\end{minipage}\hfill
	\begin{minipage}[b]{0.48\textwidth}\centering
		\includegraphics[width=1.0\textwidth]{billeder/snitfem.png} %Højre billede
	\end{minipage}\\ %Captions and labels
	\begin{minipage}[t]{0.48\textwidth}
		\caption{Fritlegemediagram for snit 1} %Venstre caption og label
		\label{fig:snitet}
	\end{minipage}\hfill
	\begin{minipage}[t]{0.48\textwidth}
		\caption{Fritlegemediagram for snit 5} %Højre caption og label
		\label{fig:snitfem}
	\end{minipage}
\end{figure}

\textbf{Snit 5: 0 m < x < 6 m}
\newline
Fritlegemediagrammet for snit 1 ses på Figur \ref{fig:snitfem}.
\newline
\newline
Normalkraften bestemmes først:
\begin{equation}
	0 = - N_5 + F_x - v_2 \cdot \SI{16}{m} - j - v_k \cdot 2m \leftrightarrow N_5 = \SI{49,57}{kN} - \SI{48,12}{kN} = \SI{1,45}{kN}
\end{equation}

Forskydningskraften for snit 5 bestemmes ved:
\begin{equation}
	0 = V_5 - E_1 \cdot \SI{18}{m} - F_y + C_y - E_s \cdot x \leftrightarrow V_5(x) = \SI{-172,35}{kN} + 1,\!24 \frac{\text{kN}}{\text{m}}x
\end{equation}

Til sidst bestemmes moment:
\begin{equation}
\begin{split}
	0 = & - M_5 - F_y \cdot x - E_1 \cdot \SI{18}{m} \cdot x - E_s \cdot x \frac{x}{2} - v_2 \cdot \SI{16}{m} \cdot \SI{8}{m} - v_k \cdot \SI{2}{m} \cdot \SI{17}{m} - j \cdot \SI{17,33}{m} + C_y \cdot x \leftrightarrow \\ & M_5(x) =  \SI{172,35}{kN} \cdot x - 0,\!62 \frac{\text{kN}}{\text{m}} \cdot x^2 - \SI{1011,72}{kNm}
\end{split}
\end{equation}

Tabel \ref{tab:resultaterbrud} viser resultaterne for alle snittene. 

\begin{table}
	\begin{center}
		\begin{tabular}{|c|c|c|c|c|c|c|}
			\hline
			Snit/Værdi & N($x_{min}$) & N($x_{max}$) & V($x_{min}$) & V($x_{max}$) & M($x_{min}$) & M($x_{max}$) 	\\ \hline
			1, 0<x<2  & -547,95       & -484,87    	&  133,31    	&  66,02 	&  0,00     &  178.00        		\\ \hline
			2, 2<x<18 &  -484,87        &  19,78       &  53,85      & -43,46   &  178,00  &  455,80    \\ \hline
			3, 0<x<6  & 1,45       &  1,45     &  -72,24         &  -79,69     &  403,78     &  0,21 			    \\ \hline
			4, 0<x<18 &  -1027,92       &  85,20      &  0,00        &  0,00    &  0,00   &   0,00    \\ \hline
			5, 0<x<6  &  1,45     &    1,45      &  -172,35      &  -164,89     &   -1011,72        &   0,00      		\\ \hline
			6, 2<x<18 &  -716,74  &   -212,09  &   -64,89    &   -45,73    &    -88,62       &   -1011,94      		\\ \hline
			7, 0<x<2 &  -779,83        &   -716,75       &     0,00      &   -67,29   &    0,00     &    -88,62     		\\ \hline
		\end{tabular}
		\caption{Snitkræfter for brudgrænsetilstand, N og V: kN og M: $kNm$}
		\label{tab:resultaterbrud}
	\end{center}
\end{table}

Ud fra snittene er der lavet snitkurver, som er illusteret på Figur \ref{fig:forskydningskurve}, Figur \ref{fig:momentkurve} og Figur \ref{fig:normalkraftkurve}. Værdierne i kurverne aflæses i Tabel \ref{tab:resultaterbrud}, hvor $V_1$ angiver værdien for forskydningskraften ved snit 1. Altså er $V_1(0m) = 133,31 kN$, $V_1(2m) = 66,02 kN$, osv.

\begin{figure}[H]
	\centering
	\includegraphics[width=0.7\textwidth]{billeder/sk.png}
	\caption{Forskydningskurve}
	\label{fig:forskydningskurve}
\end{figure}

\begin{figure}[H]
	\centering
	\includegraphics[width=0.7\textwidth]{billeder/skkm.png}
	\caption{Momentkurve}
	\label{fig:momentkurve}
\end{figure}

\begin{figure}[H]
	\centering
	\includegraphics[width=0.7\textwidth]{billeder/SKFN.png}
	\caption{Normalkraftkurve}
	\label{fig:normalkraftkurve}
\end{figure}

\section{Spænding og dimensioneringskriterier}

\begin{figure}[H]
	\centering
	\includegraphics[width=0.5\textwidth]{billeder/iprofil.png}
	\caption{I-profil 450}
	\label{fig:iprofil}
\end{figure}

Hovedformålet med dette afsnit er, at undersøge om konstruktionen har en tilstrækkelig bæreevne. Derfor undersøges spændingstilstanden ved Von Mises, hvor det skal gælde: 
\begin{equation}
	\sqrt{\sigma^2 + 3\tau^2} \le \frac{f_{yk}}{\gamma}
\end{equation}

\begin{itemize}
	\item[-] $\sigma$: Normalspænding 
	\item[-] $\tau$: Forskydningsspænding
	\item[-] $f_{yk}$: Den karakteristiske flydespænding, der afhænger af ståltype. For ståltype S235 ved $16 < t \le 40$ er $f_{yk} = 225 MPa$ \citep[ s. 213]{stabi}
	\item[-] $\gamma$: Partialkoefficient, der sættes til 1,1 \citep[ s. 212]{stabi}
\end{itemize}

Spændingstilstanden undersøges i fem snit på I-profilet, som ses på Figur \ref{fig:iprofilsnit}.

\begin{figure}[H]
	\centering
	\includegraphics[width=0.4\textwidth]{billeder/iprofilsnit.png}
	\caption{I-profil med snit}
	\label{fig:iprofilsnit}
\end{figure}

Først beregnes normal- og forskydningsspændingen. 

\subsection{Forskydningsspænding}
Forskydningsspændingen $\tau$ beregnes ved Grasshofs formel:
\begin{equation}
\tau = \frac{VQ}{Ib}
\end{equation}
\begin{itemize}
	\item[-] $V$: Forskydningskraft, som er bestemt igennem snitkræfter
	\item[-] $Q$: Statisk moment
	\item[-] $b$: Bredde
\end{itemize}

Forskydningsspændingsfordelingen på I-profilet findes ved, at lave to snit, som er illustreret på Figur \ref{fig:snitsnit}. 

\begin{figure}[H]
	\centering
	\includegraphics[width=0.3\textwidth]{billeder/forskydningprofil.png}
	\caption{Snit 1 og snit 2 på I-profilet til bestemmelse af forskydningsspændingsfordeling}
	\label{fig:snitsnit}
\end{figure}

Den ubekendte faktor til beregningen af forskydningsspændingen er det statiske moment \textit{Q}, der er defineret som:
\begin{equation}
Q = \bar{y}A
\end{equation}
\begin{itemize}
	\item[-] $\bar{y}$: Tyngdepunktskoordinat
	\item[-] A: Areal
\end{itemize}

Q beregnes for snit 1 og snit 2, hvormed forskydningsspændingen i henholdsvis flangen og kroppen kan beregnes. 
\newline
\newline
\textbf{Snit 1: Forskydningsspændingsfordeling i flangen}
\newline
Snit 1 er vist på Figur \ref{fig:snitetforskyd}, hvor 0 mm < z < 76,9 mm

\begin{figure}[H]
	\centering
	\includegraphics[width=0.2\textwidth]{billeder/snitetforskydning.png}
	\caption{Snit 1}
	\label{fig:snitetforskyd}
\end{figure}

Tyngdepunktskoordinat $\bar{y}$ samt arealet \textit{A} bestemmes:
\begin{equation}
\bar{y} = \frac{450 \text{mm} - 2 \cdot 24,\!3 \text{mm}}{2} + \frac{24,\!3 \text{mm}}{2} = 212,\!85 \text{mm}
\end{equation}
\begin{equation}
A = z \cdot \SI{24,3}{mm}
\end{equation}

Dermed bestemmes det statiske moment til:
\begin{equation}
Q(z) = \SI{212,85}{mm} \cdot (z \cdot \SI{24,3}{mm})
\end{equation}

\textit{z} sættes til 0 mm og 76,9 mm for at bestemme forskydningsspændingen på I-profilets flange:
\begin{equation}
Q_{0 \text{mm}} = \SI{212,85}{mm} \cdot (\SI{0}{mm} \cdot \SI{24,3}{mm}) = \SI{0}{mm^3}
\end{equation} 
\begin{equation}
	Q_{76,9 \text{mm}} = \SI{212,85}{mm} \cdot (\SI{76,9}{mm} \cdot \SI{24,3}{mm}) =  3,\!977 \cdot 10^5 \text{mm}^3 
\end{equation}

Dermed kan den mindste og den største forskydningsspænding bestemmes, hvor forskydningskraften, \textit{V} afhænger af, hvilket snit på konstruktionen der vælges:
\begin{equation}
	\tau_{max} = \frac{V \cdot 3,977 \cdot 10^5 \text{mm}^3}{458,\!5 \cdot 10^6 \text{mm}^4 \cdot 24,\!3 \text{mm}}
\end{equation}

\begin{equation}
	\tau_{min} = \frac{V \cdot 0 \text{mm}^3}{458,\!5 \cdot 10^6 \text{mm}^4 \cdot 24,\!3 \text{mm}}
\end{equation}

Den største og mindste forskydningsspænding for de valgte kritiske punkter er vist i Tabel \ref{tab:forskudning}, og forskydningsspændingsfordelingen i flangen er illustreret på Figur \ref{fig:forskydfordeling}.

\begin{table} [H]
	\begin{center}
		\begin{tabular}{c c c c }
			\hline
			Kritisk punkt   & Forskydningskraft, V [kN] & $\tau_{max}$ [MPa] & $\tau_{min}$[MPa] \\ \hline
			Snit 4 ved 0 m  & 0                       & 0              & 0                     \\ \hline
			Snit 5 ved 0 m  & -357,20                 & -12,75         & 0                     \\ \hline
			Snit 6 ved 18 m & -112,10                 & -4,00          & 0                     \\ \hline
		\end{tabular}
		\caption{Forskydningsspænding i flangen}
		\label{tab:forskudning}
	\end{center}
\end{table}

\textbf{Snit 2: Forskydningsspændingsfordeling i kroppen}
\newline
Snit 2 er vist på Figur \ref{fig:snittoforskyd}, hvor -200,7 mm < y < 200,7 mm

\begin{figure}[H]
	\centering
	\includegraphics[width=0.4\textwidth]{billeder/snittoforskydning.png}
	\caption{Snit 2}
	\label{fig:snittoforskyd}
\end{figure}

Som det ses på Figur \ref{fig:snittoforskyd} er I-profilet inddelt i to arealer; $A_1$ og $A_2$. Derfor beregnes det statiske moment \textit{Q} for hvert areal; $Q_1$ og $Q_2$, og den samlede Q er dermed $Q = Q_1 + Q_2$. 

Først findes Q for $A_1$:
\begin{equation}
	Q_1 = \SI{24,3}{mm} \cdot \SI{170}{mm} \cdot (\SI{225}{mm} - \frac{1}{2} \cdot \SI{24,3}{mm}) = 8,\!793 \cdot 10^5 \text{mm}^3
\end{equation}

Q for $A_2$, som er afhængig af y, bestemmes nu:
\begin{equation}
Q_2(y) = (\SI{16,2}{mm} \cdot (\SI{200,7}{mm} -y)) \cdot (\frac{200\!,7 \text{mm} -y}{2} + y)
\end{equation}

$Q_1$ er en konstant og $Q_2$ er en funktion af y. Dermed fås et samlet statisk moment \textit{Q}, som funktion af \textit{y}:
\begin{equation}
	Q(y) = 8,\!793 \cdot 10^5 \text{mm}^3 + ((\SI{16,2}{mm} \cdot (\SI{200,7}{mm} -y)) \cdot (\frac{200\!,7 \text{mm} -y}{2} + y))
\end{equation}

Forskydningsspændingen kan hermed beregnes ved:
\begin{equation}
	\tau(y) = \frac{V \cdot Q(y)}{458,\!5 \cdot 10^6 \text{mm}^4 \cdot 16,\!2 \text{mm}}
\end{equation}

Ved at sætte y til 0 og $\pm 200,7 mm$ findes forskydningsspændingerne langs I-profilets krop. Forskydningsspændingen vil altid være 0 i oversiden, dvs. ved y = 225 mm. Forskydningsspændingen for de valgte kritiske punkter ses i Tabel \ref{tab:kroppen}, og forskydningsspændingsfordelingen på I-profilet er illustreret på Figur \ref{fig:forskydfordeling}. 

\begin{table}
	\begin{center}
		\begin{tabular}{c c c c c }
			\hline
			Kritisk punkt   & Forskydningskraft, V [kN] & $\tau(0)$ [MPa] & $\tau(200,7)$[MPa] & $\tau(-200,7)$[MPa] \\ \hline
			Snit 4 ved 0 m  & 0                       & 0       & 0          & 0                  \\ \hline
			Snit 5 ved 0 m  & -357,20                 & -57,97  & -42,28     & -42,28     \\ \hline
			Snit 6 ved 18 m & -112,10                 & -18,19  & -13,27     & -13,27       \\ \hline
		\end{tabular}
		\caption{Forskydningsspænding i kroppen}
		\label{tab:kroppen}
	\end{center}
\end{table}

\begin{figure}[H]
	\centering
	\includegraphics[width=0.4\textwidth]{billeder/forskydningsfordeling.png}
	\caption{Forskydningsspændingsfordeling}
	\label{fig:forskydfordeling}
\end{figure}

\subsection{Normalspænding}
Normalspændingen $\sigma$ findes ved Naviers formel:
\begin{equation}
\sigma = \frac{N}{A} - \frac{M}{I} y
\end{equation}

\begin{itemize}
	\item[-] N: Normalkraft, som er bestemt igennem snitkræfter
	\item[-] A: Tværsnitsareal, som for stålprofil 450 er $14,\!7 \cdot 10^3 mm^2$ \citep{stabi}. 
	\item[-] M: Moment, som er bestemt igennem snitkræfter
	\item[-] y: Koordinat til snit
	\item[-] I: Inertimoment, som for stålprofil 450 er $458,\!5 \cdot 10^6 mm^4$ \citep{stabi}. 
\end{itemize} 

Her er den eneste ubekendte faktor y. Derfor fås normalspændingen som en funktion af y, hvor N og M afhænger af hvilket snit på konstruktionen der undersøges:
\begin{equation}
	\sigma(y) = \frac{N}{14,\!7 \cdot 10^3 \text{mm}^2} - \frac{M}{458,\!5 \cdot 10^6 \text{mm}^4} y
\end{equation}
Spændingen beregnes i snittene $y = 0 mm$, $y = \pm 200,\!7 mm$ og $y = \pm 225 mm$. Resultatet ses i Tabel \ref{tab:normalspanding}.

\begin{table} [H]
	\begin{center}
		\begin{tabular}{c c c c c c c c }
			\hline
			Kritisk punkt   & Normalkraft, N [kN] & Moment, M [kNm] & $\sigma(0)$[MPa] & $\sigma(200,7)$[MPa] & $\sigma(200,7)$[MPa] & $\sigma(225)$[MPa] & $\sigma(-225)$[MPa] \\ \hline
			Snit 4 ved 0 m  & -658,26           & 0                   & -44,78      & -44,78          & -44,78          & -44,78        & -44,78       \\ \hline
			Snit 5 ved 0 m  & -62,52          & -2120,82      & -4,25     & -3,32         & -5,18          & -3,21        & -5,29        \\ \hline
			Snit 6 ved 18 m & -396,94           & -2120,82        & -27,00      & -26,08         & -27,93          & -25,96        &                      \\ \hline
		\end{tabular}
		\caption{Normalspænding}
		\label{tab:normalspanding}
	\end{center}
\end{table}

\subsection{Spændingstilstand}
Spændingstilstanden er beregnet for alle snittene, og de største værdier er angivet i Tabel \ref{tab:tilstand}.
\newline
\newline
Den regningsmæssige flydespænding $f_y$ bestemmes til:
\begin{equation}
f_y = \frac{225 \text{MPa}}{1,\!1} = \SI{204,54}{MPa}
\end{equation}

Udnyttelsesgraden beregnes ved:
\begin{equation}
	\frac{\sqrt{\sigma^2 + 3\tau^2}}{\frac{f_{yk}}{\gamma}} \le 1
\end{equation}
Overskrider denne værdi 1, betyder det, at den regningsmæssige flydespænding er overskredet, og dermed har tilbygningen ikke en tilstrækkelig bæreevne. 

\begin{table} [H]
	\begin{center}
		\begin{tabular}{c c c c }
			\hline
			Kritisk punkt   & Maksimal spændingstilstand [MPa] & FORMEL!! & Udnyttelsesgrad \\ \hline
			Snit 4 ved 0 m  & 44,78                   & 204,54      & 0,22     \\ \hline
			Snit 5 ved 0 m  & 100,51                  & 204,54      & 0,49     \\ \hline
			Snit 6 ved 18 m & 41,50                   & 204,54      & 0,20     \\ \hline
		\end{tabular}
		\caption{Spændingstilstand}
		\label{tab:tilstand}
	\end{center}
\end{table}

Det ses, at den højeste spænding er 100,50 MPa med en udnyttelsesgrad på 0,49, og dermed overskrider spændingen ikke den regningsmæssige flydespænding. Det kan dermed konkluderes, at tilbygningen har en tilstrækkelig bæreevne. 