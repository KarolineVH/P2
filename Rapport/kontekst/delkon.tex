\chapter{Vækstaksens betydning}
Ovenstående afsnit har beskrevet Aalborgs kommuneplan, hvor Vækstaksen er blandt de vigtigste punkter i denne. Aalborg Kommune er i gang med realiseringen af denne, og særligt havnefronten har gennemgået en stor forandring de seneste 5-10 år. 
\newline \indent{     }  Formålet med Vækstaksen er, at styrke Aalborg som by, og der er en forventning fra Aalborg Kommune om, at denne bliver Nordjyllands Vækstdynamo, men er dette realistisk? Nedenstående afsnit vil stille spørgsmålstegn ved Vækstaksen, og diskutere, hvilken sammenhæng Strøybergs Palæ og Vækstaksen har med hinanden. I den forbindelse vil der diskuteres, hvorfor der ønskes en tilbygning, og hvorfor det netop er i dette område, det er fordelagtigt at udvide.

\section{Fokus på havnefronten}
Vækstaksen er placeret således, at den går gennem Aalborgs centrale områder for kultur, uddannelse, erhverv og miljø. Områderne er tæt befolket, og derfor ønsker Aalborg Kommune at etablere en god infrastruktur, som den kommende Letbane eksempelvis skal hjælpe med. Dermed er det fordelagtigt at udvide samt udvikle områderne i Vækstaksen, da de er drivkraften for Aalborg \citep{bedreoverblik}, og det er her, at byens liv er og fremtiden skabes. Selvom områderne i dag fungerer som drivkraft, skal der stadig udvikles i områderne, da Aalborg er en by i stor vækst, hvilket afspejles af både indbyggertal og antal virksomheder, som flytter til og skabes i Aalborg. Derfor kan Aalborg Kommune ikke blot stille sig tilfreds med de nuværende tilstande, da byen skal udvides, for at være forberedt på fremtiden; ellers er der ikke plads til udviklingen, og den vil bremses, gå helt i stå eller i værste fald gå den anden vej.
\newline \indent{     }  Udviklingen vil primært ske gennem en byfortætning, da de centrale områder allerede i dag sprudler med liv gennem kultur, erhverv og uddannelse. Der er altså ikke tale om store udvidelser af områderne, men snarere optimeringer af de allerede eksisterende områder, og derfor kan infrastrukturen opleve problemer. Derfor kan spørgsmålet stilles, om det er realistisk med en fortætning af byen samtidig med, at Aalborg Kommune har en målsætning om, at det skal være let at færdes i byen. Letbanen kan afhjælpe trafikken, men i takt med fortætningen er spørgsmålet, om den er nok, til at det føles let at færdes i Aalborg. Allerede i dag ønskes en tredje Limfjordsforbindelse for køretøjer, da trafikken til og fra Aalborg er tæt i morgentimerne og eftermiddagstimerne, og særligt når der opstår et uheld enten i Limfjordstunnelen eller omkring Limfjordsbroen, hvilke er de to eneste forbindelser for køretøjer over Limfjorden, da den anden forbindelse så bliver overbelastet. 
\newline
\newline
Aalborg har et ønske om at være Danmarks bedste studieby \citep{ungdom}, og derfor kræves der nok studieboliger til de studerende. Der er i landet en generel mangel på studieboliger, og dette gør sig også gældende for Aalborg, selvom det ikke er i ligeså høj grad som eksempelvis København. Det skyldes Aalborgs store fokus på de studerende og behovet for studieboliger, hvoraf der er blevet bygget over 6000 studieboliger siden 2010 inden for Vækstaksen, både på Aalborg og Nørresundby siden, og der bygges fortsat flere nye boliger i dag \citep{studieboliger}. Dette er et led i Vækstaksen og planerne om Aalborgs fremtid, hvilket indtil videre er godt realiseret. De seneste fem år har de videregående uddannelser oplevet rekordmange ansøgninger og dermed flere studerende, hvilket betyder, at studieboligerne hurtigt bliver lejet ud. Et af fokuspunkterne fra Aalborg Kommunes side er, at gøre Aalborg en attraktiv studieby med studievenlige priser på boligerne. Sammenlignes priserne med de tre andre storbyer i Danmark; Aarhus, Odense og København, er boligerne i Aalborg væsentlig billigere, hvilket kan være en medvirkende faktor til, at Aalborg er attraktiv for de unge. Dette punkt fra Vækstaksen er derfor realiseret.
\newline
\newline
Vækstaksen er af stor betydning for Aalborg, og denne udgør en stor del af Aalborg Kommuneplan, og dermed planerne for fremtiden. Flere af punkterne i Vækstaksen er beskrevet ovenfor, men her har fokus været på studerende samt infrastrukturen. Aalborg Kommune har også et ønske om, at byen skal være for alle mennesker i alle aldersgrupper samt et attraktivt sted for erhverv. Spørgsmålet er dog, hvor erhverv skal etableres inden for Vækstaksen, og hvad der kan gøres for de indbyggere, som ikke er studerende. Ses der på naturområderne og de mange attraktioner i byen, så er disse for alle aldersgrupper, men havnefronten, som er en af de bærende elementer i Vækstaksen, anvendes primært af unge mennesker, og der er ikke meget, der indbyder til familier med børn eller for den ældre generation. For at Vækstaksen skal få den betydning, som Aalborg Kommune ønsker, er det nødvendigt med flere tiltag for den ældre aldersgruppe. 
\newline \indent{     }  For nye virksomheder gælder det, at der skal være nok kontorlokaler og bygninger, som kan husere virksomhederne. Ligeledes skal virksomhedens interesserer gerne falde ind med Aalborg og passe sammen, således at der også kommer kunder til blandt Aalborgs befolkning. Med en målsætning om, at være en attraktiv storby med mange muligheder for virksomheder, er der derfor gode muligheder for at drive virksomhed i Aalborg, også for fremtiden. Indenfor Vækstaksens bælte ligger der utallige af virksomheder, men der er stadig både tomme erhvervslokaler og planer om udvidelser af allerede eksisterende lokaler, som vil give plads til flere virksomheder. Heriblandt skal der ske en udvidelse af Strøybergs Palæ, som har en central beliggenhed i Vækstaksen, helt nede ved havnefronten. 

\section{Hvorfor udvide Strøybergs Palæ?}
Strøybergs Palæ ligger i Vækstaksen. På Figur \ref{fig:udvikling} ses det, at Strøybergs Palæ ligger i det lilla område og er dermed i et område, hvor der er størst mulighed for udvikling inden for Vækstaksen. På grund af havnefrontens udvikling de sidste 10 år, har der været fokus på udviklingen her, og nu har Aalborg Kommune ment, at det er på tide at udvikle området ved Strøybergs Palæ, da der allerede er bygget Musikkens Hus, havnefronten, Utzon Centeret og Utzon Parken omkring området. Udviklingen omkring havnefronten er sket i takt med Vækstaksens fokus herpå. Om havnefronten havde fået en udvikling overhovedet eller i lige så høj grad, hvis tankerne og ideérne omkring Vækstaksen ikke var blevet sat i værk, kan diskuteres. Strøybergs Palæ havde måske aldrig været et relevant emne at tage op i forhold til udviklingen af Aalborg, hvis ikke bygningen havde ligget i Vækstaksen. Havde Vækstaksen ligget anderledes, således Strøybergs Palæ lå uden for Vækstaksen, var det måske aldrig kommet på tale, at lave en tilbygning hertil, men blot lade den stå som den er. Placeringen af Strøybergs Palæ midt i Vækstaksen, må formodes at have haft indflydelse på beslutningen om, at der skal laves en tilbygning, fordi bygningen skal leve op til Aalborg Kommunes forventninger og ønsker for fremtiden, og fordi de nok mener, at der er udviklingspotentiale gennem denne bygning også. Tilbygningen kan bruges til erhvervslokaler, og dette vil kunne tiltrække en eller flere nye virksomheder til området, og dermed udvikle både området og Aalborg by. Her kan spørgsmålet dog også stilles, om denne tilbygning vil få den ønskede effekt og kunne leve op til disse målsætninger. Det vil være naivt at tro, at en udvidelse på nogle få hundrede kvadratmeter vil gøre en betydende forskel for Aalborg og Vækstaksen, og det kan derfor ikke alene være grunden til tilbygningen, men snarere en af flere årsager. Strøybergs Palæ er kun en lille del af Vækstaksen, og skal derfor ikke bære Aalborgs udvikling og vækst alene.
\newline \indent{     }  Hovedpunkterne i Vækstaksen er, at Aalborg skal vokse som by, og udvikles gennem en byfortætning, hvor målet er, at flere virksomheder og indbyggere kommer til byen. En udvidelse af Strøybergs Palæ vil betyde et ekstra antal erhvervslokaler, og dette passer godt sammen med Vækstaksen. Tilbygningen betyder desuden, at Strøybergs Palæ vil få et mere harmonisk udseende ud mod vandet, da tilbygningen vil blive bygget i samme stil som resten af bygningen, og den nordlige del af bygningen nu vil komme op i ca. samme højde, som resten af bygningen. Sammen med resten af havnefronten vil Strøybergs Palæ derfor gennemgå en renovering, som løfter udseendet samt medfører at området nu har en mere ensformet og harmonisk stil. 
\newline \indent{     }  Det, at området bliver et centralt fokuspunkt for fremtiden betyder også, at området bliver mere attraktivt, idét der kommer til at ske en fortsat udvikling af området, og netop derfor kan en tilbygning være en god investering, da det giver ekstra plads og bedre forudsætninger for udviklingen.

\section{Delkonklusion}
Der er i de foregående afsnit redegjort for Aalborgs historie og baggrunden for byens udvikling. Ligeledes er der beskrevet, hvad en kommuneplan og lokalplan er, og der er lavet en dybdegående analyse af Aalborgs kommuneplan, med særligt fokus på Vækstaksen. Inden for vækstaksen ligger Strøybergs Palæ, som hører under den nuværende lokalplan 1-1-107, hvilken også er behandlet, analyseret og sammenlignet med den tidligere lokalplan for området, 10-0-82. Endelig er der foretaget en diskussion, hvor spørgsmålet stilles, hvorfor det er godt at udvikle i netop dette område, og derunder hvilken sammenhæng der er mellem Strøybergs Palæ og Vækstaksen.
\newline
\newline
Aalborg Kommune er vokset som by gennem tiden, og gennem den nuværende kommuneplan har kommunen en målsætning om, at blive Nordjyllands Vækstdynamo og blive en by med fokus på udvikling af studerende, erhverv, kultur med mere. 
\newline \indent{     }  Kommuneplanen er den overordnede ramme for byens udvikling, og den indeholder en 12-årig plan, hvor blandt andet målsætninger og planer er beskrevet. Kommuneplanen er opdelt i flere forskellige fokuspunkter, og den dækker derfor alt fra bæredygtighed og miljø til mobilitet og udvikling af samfundet. 
\newline \indent{     }  Under kommuneplanen findes lokalplanen, som beskriver et mindre område inden for kommunen og har til formål at styre udviklingen af dette. Lokalplanen skal stemme overens med kommuneplanens rammer, og der kan derfor ikke være modsigelser mellem de to planer. Opstår der komplikationer med udarbejdelsen af en lokalplan inden for en kommuneplans rammer, så anvendes der et kommuneplantillæg, for at løse problemet. De to planer er derfor tæt knyttet til hinanden, hvor kommuneplanen er den ledende af de to, men hvor der er sammenhæng mellem alle punkter i begge planer.
\newline
\newline
Vækstaksen beskriver et område i Aalborg, hvor der er planer om fremtidig udvikling. Disse områder består allerede i dag af virksomheder, uddannelsesinstitutioner, kulturattraktioner og andre interessepunkter for Aalborg Kommune. For Aalborg Kommune er det netop disse områder, som skal skabe den fremtidige vækst for Aalborg, og derfor er det fordelagtigt at udvikle yderligere i områderne. 
\newline \indent{     }  Med en beliggenhed inden for Vækstaksen, og i et af de planlagte vækstområder (se Figur \ref{fig:udvikling}), danner Strøybergs Palæ grundlag for en udvidelse. Ved at lave en tilbygning vil der blive plads til ekstra erhvervslokaler og lejligheder, og tilbygningen vil derfor være en medvirkende faktor til at styrke Aalborgs vækst.