\chapter{Delkonklusion}
Der er i de foregående afsnit redegjort for Aalborgs historie og baggrunden for byens udvikling. Ligeledes er der beskrevet, hvad en kommuneplan og lokalplan er, og der er lavet en dybdegående analyse af Aalborgs kommuneplan, med særligt fokus på Vækstaksen. Inden for vækstaksen ligger Strøybergs Palæ, som hører under den nuværende lokalplan 1-1-107, hvilken også er behandlet, analyseret og sammenlignet med den tidligere lokalplan for området, 10-0-82. Endelig er der foretaget en diskussion, hvor spørgsmålet stilles, hvorfor det er godt at udvikle i netop dette område, og derunder hvilken sammenhæng der er mellem Strøybergs Palæ og Vækstaksen.
\newline
\newline
Aalborg Kommune er vokset som by gennem tiden, og gennem den nuværende kommuneplan har kommunen en målsætning om, at blive Nordjyllands Vækstdynamo og blive en by med fokus på udvikling af studerende, erhverv, kultur med mere. 
\newline \indent{     }  Kommuneplanen er den overordnede ramme for byens udvikling, og den indeholder en 12-årig plan, hvor blandt andet målsætninger og planer er beskrevet. Kommuneplanen er opdelt i flere forskellige fokuspunkter, og den dækker derfor alt fra bæredygtighed og miljø til mobilitet og udvikling af samfundet. 
\newline \indent{     }  Under kommuneplanen findes lokalplanen, som beskriver et mindre område inden for kommunen og har til formål at styre udviklingen af dette. Lokalplanen skal stemme overens med kommuneplanens rammer, og der kan derfor ikke være modsigelser mellem de to planer. Opstår der komplikationer med udarbejdelsen af en lokalplan inden for en kommuneplans rammer, så anvendes der et kommuneplantillæg, for at løse problemet. De to planer er derfor tæt knyttet til hinanden, hvor kommuneplanen er den ledende af de to, men hvor der er sammenhæng mellem alle punkter i begge planer.
\newline
\newline
Vækstaksen beskriver et område i Aalborg, hvor der er planer om fremtidig udvikling. Disse områder består allerede i dag af virksomheder, uddannelsesinstitutioner, kulturattraktioner og andre interessepunkter for Aalborg Kommune. For Aalborg Kommune er det netop disse områder, som skal skabe den fremtidige vækst for Aalborg, og derfor er det fordelagtigt at udvikle yderligere i områderne. 
\newline \indent{     }  Med en beliggenhed inden for Vækstaksen, og i et af de planlagte vækstområder (se Figur \ref{fig:udvikling}), danner Strøybergs Palæ grundlag for en udvidelse. Ved at lave en tilbygning vil der blive plads til ekstra erhvervslokaler og lejligheder, og tilbygningen vil derfor være en medvirkende faktor til at styrke Aalborgs vækst.