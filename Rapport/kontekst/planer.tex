\chapter{Kommuneplan}
En kommuneplan er kommunens overordnede plan for kommunens udvikling. Indenfor en periode på 12 år fastlægger kommunen de overordnede mål og retningslinjer for kommunens udvikling såvel i byerne som i det åbne land. 
\newline
\newline
En kommuneplan består af; en hovedstruktur, retningslinjer, kommuneplanrammer, bilag og tilhørende planredegørelse. 
\newline \indent{     }  Hovedstrukturen er den overordnede, strategiske og sammenfattende fysiske plan for kommunen. Den fastlægger de overordnede mål for udviklingen inden for de enkelte sektorer for hele kommunen og for de enkelte områder. 
\newline \indent{     }  Retningslinjerne udgør de overordnede rammer for kommuneplanlægningen. De fastsætter principperne for arealanvendelsen i kommunen, og danner ligeledes grundlag for kommunens administration af planlovens landzonebestemmelser, samt administrationen af kompetencer indenfor anden lovgivning, herunder natur-, miljø-, bygge- og vejlovgivningen og husdyrloven. Retningslinjerne angiver sammen med områdeudpegningerne hvilke forhold, der skal tages hensyn til i administrationen, og hvilke konkrete skøn der skal foretages for disse områder. 
\newline \indent{     }  Kommuneplanrammerne styrer den overordnede arealanvendelse og danner ramme for indholdet i nye lokalplaner. Planrammerne fastlægger dermed mål, muligheder og begrænsninger for arealanvendelse i de enkelte dele af kommunen. Kommuneplanrammerne har to niveauer: 1) by/bydel/landområde og 2) rammeområder. Det første niveau “by/bydel/landområde”, behandler områdets særlige problemer, værdier og muligheder i en sammenhæng. Det andet niveau “rammeområder”, er det mest detaljerede niveau i kommuneplanen rent geografisk. Her fastsættes de bestemmelser der danner grundlag for lokalplaner.
\newline \indent{     }  Bilag er de generelle rammebestemmelser, hvor der henvises til de aktuelle bilag fra de enkelte emner.
\newline \indent{     }  Planredegørelser beskriver forudsætninger for, og ændringerne i den konkrete planlægning. Byrådet offentliggør, sammen med alle kommuneplanforslag eller med forslag til kommuneplantillæg\footnote{Opstår der problemer med et realisere en lokalplan ud fra kommuneplanen, så anvendes der et kommuneplantillæg, som er et supplement til den eksisterende kommuneplan. Denne kan justere og ændre bestemmelserne i kommuneplanen, for at gøre det muligt at realisere lokalplanen \citep{kommuneplan2009}.}, en redegørelse om planens baggrund og sammenhæng med anden planlægning. Kommuneplanen ledsages også af en planredegørelse og planstrategi, hvilken laves minimum hvert 4. år i tilknytning til kommunens budget. Denne er byrådets instrument og baner vejen for at realisere kommuneplanens mål. Her oplyses blandt andet om kommuneplanens væsentlige forudsætninger, planlægninger der er gennemført det forgangne år, det kommende års kommuneplaninitiativer samt byrådets vurdering af og strategi for udviklingen for både det kommende år (budgetåret), de kommende 4 år (overslagsårene) og en længere periode på 12 år. Desuden laves der jævnligt statusredegørelser, som giver et overordnet billede af kommunens fysiske udvikling og præsenterer de økonomiske tiltag, der knytter sig til kommunens sektorer og geografiske områder \citep{kommuneplan1}.