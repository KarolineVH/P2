\chapter{Problemformulering}

\section{Problemformulering og problemstillinger}
For at gennemføre planen om en tilbygning til Strøybergs Palæ kræves det, at lokalplanen for området stemmer overens med kommuneplanen. Ved konstruktionen af tilbygningen er det også nødvendigt at kende til geologien for området, for at bestemme hvilken type fundering, der skal benyttes. Dertil skal der anvendes en passende ståltype til dimensionering af stålprofilerne, for at tilbygningen har en tilstrækkelig bæreevne. Desuden skal der bestemmes en funderingstype til tilbygningen ud fra områdets geologi og jordbundsforhold. 

\begin{itemize} 
	\item Har Vækstaksen en betydning for tilbygningen af Strøybergs Palæ?
	\item Hvilken ståltype og stålprofil skal der anvendes ved tilbygningen til Strøybergs Palæ? 
	\item Hvilken funderingstype skal der benyttes ved området omkring Strøybergs Palæ? 
\end{itemize} 

\section{Problemafgrænsning}
Dette projekt beskriver tilbygningen af Strøybergs Palæ ud fra både et kontekstuelt og teknisk aspekt. I den kontekstuelle del lægges der vægt på beskrivelsen af Aalborg Kommuneplan samt udviklingen af områderne inden for Vækstaksen; hertil i særdeleshed området omkring havnefronten og Strøybergs Palæ, hvor Vækstaksens betydning for tilbygningen, samt tilbygningen relevans, diskuteres. Dertil beskrives og analyseres lokalplanen for området omkring Strøybergs Palæ, Lokalplan 1-1-107.
\newline \indent{     }  Den tekniske del belyser tilbygningen af Strøybergs Palæ ud fra to emner; konstruktionen af tilbygningen samt de geologiske forhold for området. I konstruktionsdelen opstilles der et statisk system for tilbygningen, som derefter dimensioneres for en række laster. 
\newline \indent{     }  Hertil beskrives de geologiske forhold, som gør sig gældende for Aalborg, hvortil der er foretaget jordbundsanalyser, for at finde frem til, hvilken type fundering der skal benyttes for tilbygningen.  