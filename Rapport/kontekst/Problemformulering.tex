\chapter{Problemformulering}

\section{Problemformulering og problemstillinger}
For at gennemføre planen om en tilbygning til Strøybergs Palæ kræves det, at lokalplanen for området stemmer overens med kommuneplanen, før at denne kan vedtages. Konstruktionen skal konstrueres ud fra denne og de fastsatte normer, som gælder for lokalplan 1-1-107, hvor bygherrer så har mulighed for at vælge det specifikke design for udseendet samt rammen for tilbygningen og dennes dimensioner, således at konstruktionen kan holde. Dertil skal der vælges en korrekt ståltype, profil og de rette dimensioner, således at konstruktionen har en tilstrækkelig bæreevne, og samtidig tilgodeser alle parter for konstruktionen. Hertil skal der også tages højde for de geologiske forhold for området for at bestemme, hvordan tilbygningen skal funderes. 


\begin{itemize} 
	\item Har Vækstaksen betydning for tilbygningen til Strøybergs Palæ, for at opnå en tilstrækkelig bæreevne?
	\item Hvilken ståltype og stålprofil skal der anvendes ved tilbygningen til Strøybergs Palæ? 
	\item Hvilken funderingstype skal der benyttes ved tilbygningen til Strøybergs Palæ og med hvilke dimensioner?
\end{itemize} 

\section{Problemafgrænsning}
Dette projekt beskriver tilbygningen til Strøybergs Palæ ud fra både et kontekstuelt og teknisk aspekt. I den kontekstuelle del lægges der vægt på beskrivelsen af Aalborg Kommuneplan samt udviklingen af områderne inden for Vækstaksen; hertil i særdeleshed området omkring Aalborg Havnefront og Strøybergs Palæ, hvor Vækstaksens betydning for tilbygningen, samt tilbygningen relevans, diskuteres. Dertil beskrives og analyseres lokalplanen for området omkring Strøybergs Palæ, Lokalplan 1-1-107.
\newline \indent{     }  Den tekniske del belyser tilbygningen af Strøybergs Palæ ud fra to emner; konstruktionen af tilbygningen samt de geologiske forhold for området. 
\newline \indent{     }  I konstruktionsdelen opstilles der et statisk system for tilbygningen, som derefter dimensioneres for en række laster. 
\newline \indent{     }  I den geotekniske del 2 beskrives de geologiske forhold, som gør sig gældende for Aalborg, hvortil der er foretaget jordbundsanalyser, for at finde frem til, hvilken type fundering der skal benyttes for tilbygningen.  