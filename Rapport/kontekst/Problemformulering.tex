\chapter{Problemformulering}

\section{Problemformulering og problemstillinger}
Aalborg får en ny cykel/gangbro på Jernbanebroen, der skal forbinde Aalborg og Nørresundby. Forberedelsesfasen har været undervejs længe, og derfor kan spørgsmålet stilles, om det er nødvendigt med en ny cykel/gangbro ved Jernbanebroen. Under forberedelsesfasen er flere løsningsforslag med forskellige dimensioner og materialer blevet fremlagt, hvor fordele og ulemper vurderes.
\begin{itemize} \item Hvad er baggrunden for at konstruere en ny cykel/gangbro på Jernbanebroen? \item Hvad er borgernes holdning til den kommende cykel/gangbro? \item Hvor meget vil cykel/gangbroen øge belastningen på de indre kræfter i brokonstruktionens stålelementer? \item Hvorfor er de udvalgte dimensioner på cykel/gangbroen at foretrække?
\end{itemize} 

\section{Problemafgrænsning}
Til dette projekt ses der både på de kontekstuelle og tekniske aspekter for konstruering af Jernbanebroen samt den nye cykel/gangbro. I den kontekstuelle del vælges der at se på historien bag Limfjordsforbindelserne, hvor der holdes et fokus på den gamle og nye Jernbanebro. Etableringsprocessen af den nye cykel/gangbro samt årsagen til udskydelsen af den undersøges. Derudover indgår et interview med formanden for organisationen Kulturbro Aalborg, som står i spidsen for projektet. Der uddybes desuden hvem cykel/gangbroen vil gavne, og der undersøges via et spørgeskema, hvorvidt der er interesse for denne.
\newline \indent{     }  I den tekniske del fokuseres der på lasterne af Jernbanebroen, idet de er relevante for cykel/gangbroen, når den skal monteres på siden af broen. Udover dette undersøges bæreevnen af først Jernbanebroen, og derefter den sammensatte bro bestående af Jernbanebroen med cykel/gangbroen monteret på siden. Løsningsforslag 1 vil blive anvendt til at beregne, om Jernbanebroen vil kunne bære en ny cykel/gangbro. Her vil der også blive fokuseret på de anvendte materialer for både Jernbanebroen og for cykel/gangbroen. De tre løsningsforslag for den nye cykel/gangbro sættes op imod hinanden. 