\chapter{Aalborg i vækst}
Aalborg har stor fokus på udviklingen af byen, og flere centrale områder i byen har indenfor de seneste fem år gennemgået en større renovering og fornyelse. Følgende afsnit vil beskrive byens udvikling helt tilbage fra dens grundlæggelse og frem til i dag, da Vækstaksen yderligere vil ændre på Aalborgs identitet som tidligere industriby. Slutteligt vil en af Aalborgs gamle og bevaringsværdige bygninger, Strøybergs Palæ, beskrives, da der er planer om en udvikling af denne bygning i form af en tilbygning. 

\section{Aalborg gennem tiden}
Aalborg er en by fra omkring år 1040, og er i dag Danmarks fjerde største by \citep{Denstore}. Byen har gennem årene udviklet sig til en købstadsby og handelscentrum for Nordjylland \citep{byhistorie}. Dette grundet dens beliggenhed ved Limfjordens smalleste punkt og udmundingen af tre større åer, der var grundlag for små havne. Beliggenheden har gjort det muligt for Aalborg, at udvikle sig i størrelse og som by. I 1500-tallet nød byen godt af silde- og korneksport til Norge, samt studeeksport til Tyskland \citep{byhistorie}. Da Limfjorden var sandet til, skulle alt eksport vest for Aalborg gå gennem Aalborg for at blive fragtet til Norge og andre steder. Aalborg fik velstand og flere indbyggere der i 1600-tallet nåede op på omkring 4.000 indbyggere \citep{indbyggertal}. På daværende tidspunkt blev Aalborg Danmarks andenstørste by. Dette varede ikke ved, da andre jyske byer blev prioriteret højere af kongen, og derved udviklede de sig mere eksponentielt i det kommende århundrede end Aalborg. Det, at Limfjorden brød igennem på vestsiden og Danmark mistede Norge, gavnede heller ikke Aalborgs position i forhold til handlen \citep{indbyggertal}.
\newline \indent{     }  Til trods for dette fortsatte Aalborg alligevel med at udvikle sig, og 1800-tallets industrialisering fik indbyggertallet til at stige stødt. Fra 1830’erne flyttede flere industrier til Aalborg, heriblandt spritfabrikken (forløberen for De Danske Spritfabrikker) og C.W Obels Tobaksfabrik. De store kridt- og lerfund i undergrunden ved Aalborg lagde grunden for cementindustrien, hvor Aalborg Portland, som den største cementfabrik, alene beskæftigede 450 ansatte. Mange flere industrier blev grundlagt eller flyttede til Aalborg, og en del af disse var blandt landets største inden for deres område. Dette fik indbyggertallet til at stige gennem en cirka 50-årig periode til 30.000. Det var ikke kun indbyggertallet der steg. For at Aalborg by skulle kunne følge med eksporten, som også dengang hovedsageligt foregik via skibsfragt, udvidede byen havnen, som i starten af 1900-tallet blev Danmarks næststørste havn \citep{byhistorie}.
\newline \indent{     }  Befolkningstallet er siden 1990'erne vokset, og er i dag stadig stigende \citep{indbyggertal}. Dette er resultatet af, at Aalborg har udviklet sig fra at være en industriby til at være en kompetenceby \citep{kommuneplan3}. Hovedårsagen til denne udvikling er, at størstedelen af industrien, som Aalborg var kendt for at have inde i centrum af byen, og som var hoveddelen af identiteten af Aalborg, er flyttet til yderkanterne af byen. I stedet er der nu fokus på at have et levende Aalborg, som skal kendes for at være en kompetenceby \citep{kommuneplan3}.
\newline \indent{     }  Havnefronten på Aalborgs side af Limfjorden er blevet et nøglepunkt for Aalborg Kommune, da denne tidligere var fyldt med fabrikker. Der er siden år 2000 foretaget mange ændringer ved havnefronten for at gøre Aalborg mere attraktiv og give den en ny identitet som byen for kompetence og innovation \citep{brughavnen}.
\newline \indent{     }  Disse ændringer har blandt andet været, at dele af den industri, som lå ved havnefronten, er flyttet ud af det centrale Aalborg og ud til mere fremkommelige steder \citep{kommuneplan3}. Dette har givet plads til et område med mere serviceerhverv. Denne type erhvervsområde gør det også muligt at opføre boliger i området, da der ikke vil være de samme støjgener, som der ville komme fra et industriområde. Denne udvikling fik muligheden for at tage fart, efter der blev lavet en ændring i planloven den 1. juli 2003, der ændrede bestemmelserne for byomdannelse. Denne ændring gjorde det muligt for kommunerne at omdanne tidligere industriområder til blandt andet servicebygninger og boligområder \citep{sort}.
En del af denne udvikling har været bebyggelsen af flere forskellige kollegieboliger som for eksempel Bikuberne ved Utzon Parken og Larsen Waterfront. Disse boliger ligger tæt op ad nogle nye kulturelle bygninger som Utzon Centeret og Musikkens Hus, samt Nordkraft, som tidligere var et kulkraftværk, men i dag er omdannet til et kulturhus med mulighed for både sport og kulturelle oplevelser. Disse byggerier viser den udvikling, som Aalborgs centrale havnefront gennemgår fra industriområde til boligområde. Havnefronten er i en stadig udvikling, da der fortsat kommer flere boligbyggerier til. Det har også givet anledning til tilbygninger, her i blandt en kommende tilbygning til den bevaringsværdige bygning Strøybergs Palæ, der ligger ved havnefronten \citep{havnefronterne}.

\section{Strøybergs Palæ}
Lokalplan 1-1-107 har opdelt området, omfattende Strøybergs Palæ, i to delområder. Delområde A med matrikelnummer 518g omfatter den bevaringsværdige sidebygning til Strøybergs Palæ, og delområde B med matrikelnummer 519b omfatter den bevaringsværdige hovedbygning til Strøybergs Palæ \citep[ s. 7]{lokalplan}. Hovedbygningen ligger på Gammel Havn 1, og er fordelt på stueetage, 1. sal, 2. sal samt kælder og tagetage, mens sidebygningen ligger på Nyhavnsgade 9 og er fordelt på stueetage, 1. sal, 2. sal og 3. sal samt kælder og tagetage. Ejendommen er op til 22 m høj og har et samlede grundareal på 1037 kvm \citep{byggesagen}.
\newline \indent{     }  Strøybergs Palæ er siden opførelsen blevet anvendt til mange forskellige formål. Nyhavnsgade 9 blev opført som en ejerlejlighedsejendom, opdelt i ni ejerlejligheder af forskellig størrelse, og Gammel Havn 1 blev på daværende tidspunkt primært brugt til erhverv. Kælderen blev i år 1920 omdannet fra hestestald til garage, og er derudover blevet brugt som lager \citep{byggesagen}. I dag bliver Strøybergs Palæ hovedsageligt anvendt til erhverv, og huser blandt andet ejendomsmæglerfirmaet EDC Danebo.
\newline \indent{     }  Siden år 2010 har det været på tale, at lave en tilbygning til Strøybergs Palæ, og omdanne bygningen til lejligheder med udsigt over Limfjorden \citep{Calum}. I den forbindelse blev lokalplan 1-1-107 udarbejdet med et ønske om at lave denne tilbygning. Området ønskes hovedsageligt anvendt til kontor- og serviceerhverv samt boligformål \citep[ s. 7]{lokalplan}.