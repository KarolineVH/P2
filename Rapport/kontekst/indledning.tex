\chapter{Indledning}
Aalborg Kommune er med et indbyggertal på over 200.000 og et areal på cirka  $1,\!140 km^2$ en af landets største kommuner \citep{kommunedata}. Det område, som Aalborg Kommunen dækker, er vist på Figur \ref{fig:aalborgkommune}. 

\begin{figure}[htbp]
	\includegraphics[width=1.0\textwidth]{billeder/aalborgkommune.png}
	\caption{Aalborg Kommune}
	\label{fig:aalborgkommune}
\end{figure}

Aalborg er en tidligere industriby, og førhen var industriområderne placeret i Aalborg Centrum. Nye planer for Aalborg har ført denne industri ud i nogle yderpunkter af Aalborg by. Derfor har kommunen nye visioner om at flytte kultur, studieliv og turisme ind, hvor der før var industri. Kommunens visioner omkring byens udvikling fremgår af kommuneplanen, hvor det primære fokus i denne rapport vil være et område gennem Aalborg, der er tiltænkt meget vækst; dette betegnes som Vækstaksen.
\newline \indent{     }  Indenfor de seneste 10 år har Aalborgs havnefront gennemgået en stor udvikling. Denne er stadig i gang, hvilket ses ved, at der kommer flere boligbyggerier til havnen, som Strøybergs Palæ er en del af. 
\newline \indent{     }  Det har siden år 2010 været på tale at lave en tilbygning til den bevaringsværdige bygning Strøybergs Palæ. Hovedbygningen af Strøybergs Palæ er fra år 1900 \citep{hovedbygning}, mens sidebygningen er fra år 1908 \citep{sidebygning}. Strøybergs Palæ er beliggende centralt i Aalborg ved Slotspladsen, der er en ca. 200 m vejstrækning ved Aalborg Havnefront, samt i nærheden af museet Utzon Centret og overfor shoppingcentret Friis. Figur \ref{fig:aalborg} viser Strøybergs Palæs beliggenhed i Aalborg.

\begin{figure}[htbp]
	\centering
	\includegraphics[width=1.0\textwidth]{billeder/aalborg.png}
	\caption{Strøybergs Palæs beliggenhed}
	\label{fig:aalborg}
\end{figure}

Når der skal anlægges nye arealer, bygninger, veje osv., skal det opføres i henhold til en lokalplan, der dækker et mindre område inden for kommunen, og har til formål at styre udviklingen inden for dette område ved hjælp af fastlagte regler og målsætninger. Den gældende lokalplan for området ved Strøybergs Palæ er lokalplan 1-1-107.
\newline
\newline
Med udgangspunkt i lokalplanen skal tilbygningen dimensioneres; herunder dets stålprofiler og fundament. 
\newline \indent{     }  Tilbygningens mål bestemmes ud fra lokalplanen og der vælges en ståltype og stålprofil. Herudfra vurderes det, om konstruktionen har en tilstrækkelig bæreevne. Brud- og anvendelsesgrænsetilstanden beregnes, da disse er dimensionsgivende for konstruktionen. Brudgrænsetilstanden giver et mål for, om der opstår brud i konstruktionens stålrammer, og anvendelsesgrænsetilstanden giver et mål for, hvor stor udbøjning stængerne oplever ved lasterne.
\newline \indent{     }  Før der kan foretages en dimensionering af fundamentet, skal det besluttes, hvilken funderingstype der skal bruges. Derfor er de geologiske forhold for området omkring Strøybergs Palæ vigtige at kende til, før en tilbygning kan udføres, da konstruktionen skal bæres af jordlagene under bygningen. I denne rapport ønskes en bestemt funderingstype; direkte fundering. Til bestemmelse af fundamentet ønskes friktionsvinklen bestemt, hvilket vil gøres ud fra fire laboratorieforsøg, som vil blive beskrevet i rapporten.